% Copyright 2020 by an anonymous marmot
%
% This file may be distributed and/or modified
%
% 1. under the LaTeX Project Public License and/or
% 2. under the GNU Public License.
%
% See the file doc/generic/pgf/licenses/LICENSE for more details.
\ProvidesFileRCS{tikzlibrary3dtools.code.tex}
\usetikzlibrary{3d,decorations,fpu}% to do: ability to switch on and off
\usepackage{calculator}% maybe drop
\makeatletter%
\ifnum\month>10\relax
\typeout{Hibernation season has just started.}%
\fi
\ifnum\month<5\relax
\typeout{Hibernation season has not yet ended.}%
\fi
\ifnum\month=2\relax
\ifnum\day=2
\typeout{Yay! Groundhog Day!}%
\fi
\fi
\def\tikz@td@extractpgfversionaux#1.#2.#3|#4#5#6;{\def#4{#1}\def#5{#2}\def#6{#3}}%
\def\tikz@td@checkversion{%
\edef\pgfutil@tmp{\noexpand\tikz@td@extractpgfversionaux\pgfversion|\noexpand\pgfutil@tmpa\noexpand\pgfutil@tmpb\noexpand\pgfutil@tmpc;}%
\pgfutil@tmp%\typeout{a=\pgfutil@tmpa,b=\pgfutil@tmpb,c=\pgfutil@tmpc}%
\pgfmathtruncatemacro{\pgfutil@tmpe}{ifthenelse(10*\pgfutil@tmpa+\pgfutil@tmpb<31,0,1)}%
\ifnum\pgfutil@tmpe=0%
\message{You are using a too old version of pgf (\pgfversion). You need at least
version 3.1.1 to use the features of this library.}%
\fi
\edef\pgfutil@tmpd{\expandafter\@firstoftwo\pgfutil@tmpc\empty}%
\pgfmathtruncatemacro{\pgfutil@tmpe}{ifthenelse(100*\pgfutil@tmpa+10*\pgfutil@tmpb+\pgfutil@tmpd<313,0,1)}%
\ifnum\pgfutil@tmpe=0%
\message{Some routines of 3dtools may not work because they rely on the pushing
and popping techniques that were added in version 3.1.3 (but the current version
is \pgfversion).^^J}%
\fi
\pgfmathtruncatemacro{\pgfutil@tmpe}{ifthenelse(100*\pgfutil@tmpa+10*\pgfutil@tmpb+\pgfutil@tmpd<316,0,1)}%
\ifnum\pgfutil@tmpe=0%
\def\pgfutil@pushmacro##1{%
    \xdef\pgfutil@pushmacro@string{\string##1}%
    \ifcsname pgfutil@pushedmacro@\pgfutil@pushmacro@string\endcsname\else
        % \newcount is \outer in Plain
        \csname newcount\expandafter\endcsname\csname pgfutil@pushedmacro@\pgfutil@pushmacro@string\endcsname
    \fi
    \global\advance\csname pgfutil@pushedmacro@\pgfutil@pushmacro@string\endcsname 1\relax
    \global\expandafter\let\csname\the\csname pgfutil@pushedmacro@\pgfutil@pushmacro@string\endcsname\pgfutil@pushmacro@string\endcsname##1%
}
\message{pgfutil@pushmacro fixed because current pgf version \pgfversion\space is smaller than 3.1.6.^^J}%
\fi
}%
\tikz@td@checkversion%
%  cos(#1)*cos(#2) 
%  & sin(#1)*cos(#2) 
%  & -sin(#2) \\
%  cos(#1)*sin(#2)*sin(#3)-sin(#1)*cos(#3) 
%  &  sin(#1)*sin(#2)*sin(#3)+cos(#1)*cos(#3) 
%  & cos(#2)*sin(#3) \\
%  cos(#1)*sin(#2)*cos(#3)+sin(#1)*sin(#3) 
%  & sin(#1)*sin(#2)*cos(#3)-cos(#1)*sin(#3) 
%  & cos(#2)*cos(#3) \\ %<-normal on screen
% 
\newcommand{\orthmat}[3]{% the entries of this matrix keep track
\pgfmathparse{cos(#1)*cos(#2)}% of the current transformation
\xdef\tikz@td@matAA{\pgfmathresult}%
\pgfmathparse{sin(#1)*cos(#2)}% 
\xdef\tikz@td@matAB{\pgfmathresult}% 
\pgfmathparse{-sin(#2)}% 
\xdef\tikz@td@matAC{\pgfmathresult}%
\pgfmathparse{cos(#1)*sin(#2)*sin(#3)-sin(#1)*cos(#3)}%
\xdef\tikz@td@matBA{\pgfmathresult}%  
\pgfmathparse{sin(#1)*sin(#2)*sin(#3)+cos(#1)*cos(#3)}% 
\xdef\tikz@td@matBB{\pgfmathresult}%
\pgfmathparse{cos(#2)*sin(#3)}%
\xdef\tikz@td@matBC{\pgfmathresult}%
\pgfmathparse{cos(#1)*sin(#2)*cos(#3)+sin(#1)*sin(#3) }%
\xdef\tikz@td@matCA{\pgfmathresult}% 
\pgfmathparse{sin(#1)*sin(#2)*cos(#3)-cos(#1)*sin(#3)}% 
\xdef\tikz@td@matCB{\pgfmathresult}%
\pgfmathparse{cos(#2)*cos(#3)}%
\xdef\tikz@td@matCC{\pgfmathresult}}%
\tikzset{3d/.cd,phi/.initial=0,psi/.initial=0,theta/.initial=0,
install view/.style={/utils/exec=\tikzset{3d/.cd,#1}%
\orthmat{\pgfkeysvalueof{/tikz/3d/phi}}{%
\pgfkeysvalueof{/tikz/3d/psi}}{\pgfkeysvalueof{/tikz/3d/theta}},%
/tikz/x={({\tikz@td@matAA*1cm},{\tikz@td@matBA*1cm})},%
/tikz/y={({\tikz@td@matAB*1cm},{\tikz@td@matBB*1cm})},%
/tikz/z={({\tikz@td@matAC*1cm},{\tikz@td@matBC*1cm})}}}%
\def\pgfmathparse@td@FPU#1{\begingroup%
\pgfkeys{/pgf/fpu,/pgf/fpu/output format=fixed}%
\pgfmathparse{#1}%
\pgfmathsmuggle\pgfmathresult\endgroup}%
\def\pgfmathsetmacro@td@FPU#1#2{\begingroup%
\pgfkeys{/pgf/fpu,/pgf/fpu/output format=fixed}%
\pgfmathsetmacro#1{#2}%
\pgfmathsmuggle#1\endgroup}%
%
\xdef\tikz@td@type{0}%0=linear combination,1=vector product
\long\def\RawCoord(#1){\csname tikz@dcl@coord@#1\endcsname}% 
\long\def\ParseCoord(#1){%
\pgfutil@tempcnta=0%
\pgfutil@for\pgf@tmp:={#1}\do{\advance\pgfutil@tempcnta by1}%
\ifnum\pgfutil@tempcnta=1\relax
\edef\pgfutil@tmp{\csname tikz@dcl@coord@#1\endcsname}% 
\fi
\ifcase\pgfutil@tempcnta
\or
\edef\pgfutil@tmp{\csname tikz@dcl@coord@#1\endcsname}% 
\or
\edef\pgfutil@tmp{#1,0}%
\or
\edef\pgfutil@tmp{#1}%
\fi%
\pgfmathparse@td@FPU{xcomp3(\pgfutil@tmp)}% 
\edef\pgf@X{\pgfmathresult pt}% 
\pgfmathparse@td@FPU{ycomp3(\pgfutil@tmp)}% 
\edef\pgf@Y{\pgfmathresult pt}% 
\pgfmathparse@td@FPU{zcomp3(\pgfutil@tmp)}% 
\edef\pgf@Z{\pgfmathresult pt}}%
\pgfmathdeclarefunction{TD}{1}{%
\begingroup%
\pgfmathtdparse{#1}%
\pgfmathsmuggle\pgfmathresult\endgroup%
}%
% projections 
\pgfmathdeclarefunction{xcomp3}{3}{% x component of a 3-vector 
\begingroup% 
\pgfmathparse@td@FPU{#1}% 
\pgfmathsmuggle\pgfmathresult\endgroup} 
\pgfmathdeclarefunction{ycomp3}{3}{% y component of a 3-vector 
\begingroup% 
\pgfmathparse@td@FPU{#2}% 
\pgfmathsmuggle\pgfmathresult\endgroup} 
\pgfmathdeclarefunction{zcomp3}{3}{% z component of a 3-vector 
\begingroup% 
\pgfmathparse@td@FPU{#3}% 
\pgfmathsmuggle\pgfmathresult\endgroup} 
\pgfmathdeclarefunction{TDx}{1}{% x component of a 3-vector 
\begingroup% 
\edef\mycoord{\RawCoord(#1)}%
\pgfmathparse@td@FPU{xcomp3\mycoord}% 
\pgfmathsmuggle\pgfmathresult\endgroup}%
\pgfmathdeclarefunction{TDy}{1}{% x component of a 3-vector 
\begingroup% 
\edef\mycoord{\RawCoord(#1)}%
\pgfmathparse@td@FPU{ycomp3\mycoord}% 
\pgfmathsmuggle\pgfmathresult\endgroup}%
\pgfmathdeclarefunction{TDz}{1}{% x component of a 3-vector 
\begingroup% 
\edef\mycoord{\RawCoord(#1)}%
\pgfmathparse@td@FPU{zcomp3\mycoord}% 
\pgfmathsmuggle\pgfmathresult\endgroup}%
% \pgfmathdeclarefunction{TDX}{1}{% x component of a 3-vector 
% \begingroup% 
% \edef\pgfmathresult{0}%
% % \typeout{#1}
% % \edef\temp{\noexpand\pgfmathparse@td@FPU{TDx#1}}%
% % \temp
% \typeout{\pgfmathresult}%
% \pgfmathsmuggle\pgfmathresult\endgroup}%
% \pgfmathdeclarefunction{TDY}{1}{% x component of a 3-vector 
% \begingroup% 
% \edef\mycoord{\RawCoord#1}%
% \pgfmathparse@td@FPU{ycomp3\mycoord}% 
% \pgfmathsmuggle\pgfmathresult\endgroup}%
% \pgfmathdeclarefunction{TDZ}{1}{% x component of a 3-vector 
% \begingroup% 
% \edef\mycoord{\RawCoord#1}%
% \pgfmathparse@td@FPU{zcomp3\mycoord}% 
% \pgfmathsmuggle\pgfmathresult\endgroup}%
\tikzset{3d/record physical components/.code={%
\tikzset{every coordinate node/.append style={3d/store components}}},
3d/store components/.code={%
\edef\pgf@Z{\relax}%
\edef\pgfutil@tmpz{\relax}%
\expandafter\ParseCoord\tikz@scan@point@coordinate%
\pgfmathsetmacro\pgfutil@tmpx{\the\pgf@x/1cm}%
\pgfmathsetmacro\pgfutil@tmpy{\the\pgf@y/1cm}%
\ifx\pgf@Z\pgfutil@tmpz
\pgfmathsetmacro\pgfutil@tmpz{screendepth(\pgfutil@tmpx,\pgfutil@tmpy,0)}%
\else
\pgfmathsetmacro\pgfutil@tmpz{screendepth(\pgf@X,\pgf@Y,\pgf@Z)}%
\fi
\expandafter\xdef\csname tikz@td@physical@x@\tikz@fig@name\endcsname{\pgfutil@tmpx}%
\expandafter\xdef\csname tikz@td@physical@y@\tikz@fig@name\endcsname{\pgfutil@tmpy}%
\expandafter\xdef\csname tikz@td@physical@z@\tikz@fig@name\endcsname{\pgfutil@tmpz}%
}}
\pgfmathdeclarefunction{TDphysx}{1}{%
\begingroup%
\ifcsname tikz@td@physical@x@#1\endcsname
\expandafter\edef\expandafter\pgfmathresult\expandafter{\csname tikz@td@physical@x@#1\endcsname}%
\else
\PackageWarning{3dtools}{Physical x component of #1 not stored. Setting it to 0.}
\edef\pgfmathresult{0}%
\fi
\pgfmathsmuggle\pgfmathresult\endgroup%
}%
\pgfmathdeclarefunction{TDphysy}{1}{%
\begingroup%
\ifcsname tikz@td@physical@y@#1\endcsname
\expandafter\edef\expandafter\pgfmathresult\expandafter{\csname tikz@td@physical@y@#1\endcsname}%
\else
\PackageWarning{3dtools}{Physical y component of #1 not stored. Setting it to 0.}
\edef\pgfmathresult{0}%
\fi
\pgfmathsmuggle\pgfmathresult\endgroup%
}%
\pgfmathdeclarefunction{TDphysz}{1}{%
\begingroup%
\ifcsname tikz@td@physical@z@#1\endcsname
\expandafter\edef\expandafter\pgfmathresult\expandafter{\csname tikz@td@physical@z@#1\endcsname}%
\else
\PackageWarning{3dtools}{Physical y component of #1 not stored. Setting it to 0.}
\edef\pgfmathresult{0}%
\fi
\pgfmathsmuggle\pgfmathresult\endgroup%
}%
\pgfmathdeclarefunction{TDphys}{1}{%
\begingroup%
\pgfmathsetmacro{\pgfutil@tmpx}{TDphysx("#1")}%
\pgfmathsetmacro{\pgfutil@tmpy}{TDphysy("#1")}%
\pgfmathsetmacro{\pgfutil@tmpz}{TDphysz("#1")}%
\edef\pgfmathresult{\pgfutil@tmpx,\pgfutil@tmpy,\pgfutil@tmpz}%
\pgfmathsmuggle\pgfmathresult\endgroup%
}%
% \def\tizk@td@scalprod#1=#2.#3;{% 
% \edef\coordA{\RawCoord#2}% 
% \edef\coordB{\RawCoord#3}% 
% \pgfmathsetmacro\pgfutil@tmpa{scalarproduct({\coordA},{\coordB})}% 
% \edef#1{\pgfutil@tmpa}}% 
\def\tizk@td@spaux#1#2#3#4#5#6{(#1)*(#4)+(#2)*(#5)+(#3)*(#6)}% 
\pgfmathdeclarefunction{scalarproduct}{2}{% scalar product of two 3-vectors 
\begingroup% 
\pgfmathparse@td@FPU{\tizk@td@spaux#1#2}% 
\pgfmathsmuggle\pgfmathresult\endgroup} 
% vector product 
% vector product auxiliary functions 
\def\vpauxx#1#2#3#4#5#6{(#2)*(#6)-(#3)*(#5)}% 
\def\vpauxy#1#2#3#4#5#6{(#4)*(#3)-(#1)*(#6)}% 
\def\vpauxz#1#2#3#4#5#6{(#1)*(#5)-(#2)*(#4)}% 
% vector product pgf functions 
\pgfmathdeclarefunction{vpx}{2}{% x component of vector product 
\begingroup% 
\pgfmathparse@td@FPU{\vpauxx#1#2}% 
\pgfmathsmuggle\pgfmathresult\endgroup} 
\pgfmathdeclarefunction{vpy}{2}{% y component of vector product 
\begingroup% 
\pgfmathparse@td@FPU{\vpauxy#1#2}% 
\pgfmathsmuggle\pgfmathresult\endgroup} 
\pgfmathdeclarefunction{vpz}{2}{% z component of vector product 
\begingroup% 
\pgfmathparse@td@FPU{\vpauxz#1#2}% 
\pgfmathsmuggle\pgfmathresult\endgroup} 
%
% axisangles, usage \pgfmathsetmacro{\myaxisangles}{axisangles("(myn)")}
\pgfmathdeclarefunction{axisangles}{1}{% 
\begingroup% 
\pgfmathsetmacro{\pgfutil@tmpa}{sqrt(TD("#1o#1"))}%
\edef\pgfutil@tmpb{\RawCoord#1}%
\pgfmathsetmacro{\pgftemp@x}{xcomp3(\pgfutil@tmpb)/\pgfutil@tmpa}% 
\pgfmathsetmacro{\pgftemp@y}{ycomp3(\pgfutil@tmpb)/\pgfutil@tmpa}% 
\pgfmathsetmacro{\pgftemp@z}{zcomp3(\pgfutil@tmpb)/\pgfutil@tmpa}% 
\pgfmathtruncatemacro{\itest}{ifthenelse(abs(\pgftemp@z)==1,0,1)}%
\ifnum\itest=0%
	\pgfmathtruncatemacro{\jtest}{sign(\pgftemp@x)}%
	\ifnum\jtest=1%
		\edef\pgfutil@tmpc{0,0}%
	\else
		\edef\pgfutil@tmpc{180,0}%
	\fi	
\else
    % 1 1
	\pgfmathsetmacro{\pgfutil@tmpb}{acos(\pgftemp@z)}% 
    \pgfmathsetmacro{\pgfutil@tmpa}{acos(\pgftemp@x/sin(\pgfutil@tmpb))}% 
    \pgfmathsetmacro{\ntest}{abs(cos(\pgfutil@tmpa)*sin(\pgfutil@tmpb)-\pgftemp@x)%
		+abs(sin(\pgfutil@tmpa)*sin(\pgfutil@tmpb)-\pgftemp@y)+abs(cos(\pgfutil@tmpb)-\pgftemp@z)}%
    \ifdim\ntest pt<0.1pt
      \edef\pgfutil@tmpc{\pgfutil@tmpa,\pgfutil@tmpb}%
    \fi
	% 1 -1
	\pgfmathsetmacro{\pgfutil@tmpb}{acos(\pgftemp@z)}% 
    \pgfmathsetmacro{\pgfutil@tmpa}{-acos(\pgftemp@x/sin(\pgfutil@tmpb))}% 
    \pgfmathsetmacro{\ntest}{abs(cos(\pgfutil@tmpa)*sin(\pgfutil@tmpb)-\pgftemp@x)%
		+abs(sin(\pgfutil@tmpa)*sin(\pgfutil@tmpb)-\pgftemp@y)+abs(cos(\pgfutil@tmpb)-\pgftemp@z)}%
    \ifdim\ntest pt<0.1pt
      \edef\pgfutil@tmpc{\pgfutil@tmpa,\pgfutil@tmpb}%
    \fi
	% -1 1
	\pgfmathsetmacro{\pgfutil@tmpb}{-acos(\pgftemp@z)}% 
    \pgfmathsetmacro{\pgfutil@tmpa}{acos(\pgftemp@x/sin(\pgfutil@tmpb))}% 
    \pgfmathsetmacro{\ntest}{abs(cos(\pgfutil@tmpa)*sin(\pgfutil@tmpb)-\pgftemp@x)%
		+abs(sin(\pgfutil@tmpa)*sin(\pgfutil@tmpb)-\pgftemp@y)+abs(cos(\pgfutil@tmpb)-\pgftemp@z)}%
    \ifdim\ntest pt<0.1pt
      \edef\pgfutil@tmpc{\pgfutil@tmpa,\pgfutil@tmpb}%
    \fi
	% -1 -1
	\pgfmathsetmacro{\pgfutil@tmpb}{-acos(\pgftemp@z)}%
    \pgfmathsetmacro{\pgfutil@tmpa}{-acos(\pgftemp@x/sin(\pgfutil@tmpb))}% 
    \pgfmathsetmacro{\ntest}{abs(cos(\pgfutil@tmpa)*sin(\pgfutil@tmpb)-\pgftemp@x)%
		+abs(sin(\pgfutil@tmpa)*sin(\pgfutil@tmpb)-\pgftemp@y)+abs(cos(\pgfutil@tmpb)-\pgftemp@z)}%
    \ifdim\ntest pt<0.1pt
      \edef\pgfutil@tmpc{\pgfutil@tmpa,\pgfutil@tmpb}%
    \fi
	%
\fi
\edef\pgfmathresult{\pgfutil@tmpc}%
\pgfmathsmuggle\pgfmathresult\endgroup} 
% projection of a 3d coordinate on the normal of the screen
\pgfmathdeclarefunction{screendepth}{3}{%
\begingroup%
\pgfkeys{/pgf/fpu,/pgf/fpu/output format=fixed}%
\pgfmathparse{%
((\the\pgf@yx/1cm)*(\the\pgf@zy/1cm)-(\the\pgf@yy/1cm)*(\the\pgf@zx/1cm))*(#1)+%
((\the\pgf@zx/1cm)*(\the\pgf@xy/1cm)-(\the\pgf@xx/1cm)*(\the\pgf@zy/1cm))*(#2)+%
((\the\pgf@xx/1cm)*(\the\pgf@yy/1cm)-(\the\pgf@yx/1cm)*(\the\pgf@xy/1cm))*(#3)}%
\pgfmathsmuggle\pgfmathresult\endgroup%
}%
% returns the x, y and z coordinates of the normal on the screen
\pgfmathdeclarefunction*{nscreenx}{0}{%
\begingroup%
\pgfkeys{/pgf/fpu,/pgf/fpu/output format=fixed}%
\pgfmathparse{((\the\pgf@yx/1cm)*(\the\pgf@zy/1cm)-(\the\pgf@yy/1cm)*(\the\pgf@zx/1cm))}%
\pgfmathsmuggle\pgfmathresult\endgroup%
}%
\pgfmathdeclarefunction*{nscreeny}{0}{%
\begingroup%
\pgfkeys{/pgf/fpu,/pgf/fpu/output format=fixed}%
\pgfmathparse{((\the\pgf@zx/1cm)*(\the\pgf@xy/1cm)-(\the\pgf@xx/1cm)*(\the\pgf@zy/1cm))}%
\pgfmathsmuggle\pgfmathresult\endgroup%
}%
\pgfmathdeclarefunction*{nscreenz}{0}{%
\begingroup%
\pgfkeys{/pgf/fpu,/pgf/fpu/output format=fixed}%
\pgfmathparse{((\the\pgf@xx/1cm)*(\the\pgf@yy/1cm)-(\the\pgf@yx/1cm)*(\the\pgf@xy/1cm))}%
\pgfmathsmuggle\pgfmathresult\endgroup%
}%
% returns a vector that is orthogonal to the input vector
\pgfmathdeclarefunction{normalcandidate}{3}{%
\begingroup%
\pgfmathtruncatemacro{\icase}{(abs(#1)>0)+2*(abs(#2)>0)+4*(abs(#3)>0)}%
\ifcase\icase
\message{All three coordinates are zero. No normal can be computed.^^J}%
\edef\pgfmathresult{0,0,0}%
\or
\edef\pgfmathresult{0,1,0}%
\or
\edef\pgfmathresult{0,0,1}%
\or
\edef\pgfmathresult{0,0,1}%
\or
\edef\pgfmathresult{1,0,0}%
\or
\edef\pgfmathresult{0,1,0}%
\or
\edef\pgfmathresult{1,0,0}%
\or
\pgfmathsetmacro{\normalization}{(#1)*(#1)+(#2)*(#2)}%
\pgfmathsetmacro{\nx}{-1*(#2)/sqrt(\normalization)}%
\pgfmathsetmacro{\ny}{#1/sqrt(\normalization)}%
\edef\pgfmathresult{\nx,\ny,0}%
\fi
\pgfmathsmuggle\pgfmathresult\endgroup%
}%
\pgfmathdeclarefunction{randompermutation}{1}{%
\begingroup%	      
\c@pgf@counta=0%
\edef\pgfutil@tmpa{0}%
\loop
\advance\c@pgf@counta by1%
\edef\pgfutil@tmpa{\pgfutil@tmpa,\the\c@pgf@counta}%
\ifnum\c@pgf@counta<#1\relax
\repeat
\loop
\advance\c@pgf@counta by-1%
\pgfmathtruncatemacro{\pgfutil@tmpb}{1+rnd*\c@pgf@counta}%
\pgfmathtruncatemacro{\pgfutil@tmpc}{{\pgfutil@tmpa}[\pgfutil@tmpb]}%
\ifnum\c@pgf@counta=\numexpr#1-1\relax
\edef\pgfmathresult{\pgfutil@tmpc}%
\else
\edef\pgfmathresult{\pgfmathresult,\pgfutil@tmpc}%
\fi
\edef\pgfutil@tmpd{\pgfutil@tmpa}%
\edef\pgfutil@tmpa{0}%
\pgfutil@for\my@item:={\pgfutil@tmpd}\do{%
\unless\ifnum\pgfutil@tmpc=\my@item
\unless\ifnum\my@item=0\relax
\edef\pgfutil@tmpa{\pgfutil@tmpa,\my@item}%
\fi\fi}%
\ifnum\c@pgf@counta>0\relax
\repeat
\pgfmath@smuggleone\pgfmathresult%
\endgroup}
\pgfmathdeclarefunction{take}{2}{%
\begingroup%
\c@pgf@counta=0%
\pgfutil@for\my@item:={#1}\do{%
\advance\c@pgf@counta by1%
\ifnum\c@pgf@counta<#2\relax
\ifnum\c@pgf@counta=1\relax
\edef\pgfmathresult{\my@item}%
\else
\edef\pgfmathresult{\pgfmathresult,\my@item}%
\fi\fi}%
\pgfmath@smuggleone\pgfmathresult%
\endgroup}
%
%
% barycenter of a list of coordinates
\pgfmathdeclarefunction{barycenter}{1}{% barycentric coordinate 
\begingroup%
\foreach \pgfutil@tmpa [count=\pgfutil@tmpb] in {#1}
{\ifnum\pgfutil@tmpb=1
   \xdef\pgfutil@tmpc{\pgfutil@tmpa}%
   \else
   \xdef\pgfutil@tmpc{\pgfutil@tmpc+\pgfutil@tmpa}%
   \fi%
   \xdef\pgfutil@tmpb{\pgfutil@tmpb}}%
\pgfmathparse{TD("\pgfutil@tmpc")}%
\pgfmathsetmacro{\pgfutil@tmpc}{1/\pgfutil@tmpb}%
\pgfmathparse{TD("\pgfutil@tmpc*(\pgfmathresult)")}%   
\pgfmathsmuggle\pgfmathresult\endgroup}%
%
% position of a real in a list of reals
\pgfmathdeclarefunction*{pos}{2}{%
\begingroup%
\c@pgf@counta=1%
\pgfutil@for\my@item:={#2}\do{%
\ifdim\my@item pt<#1pt%
\advance\c@pgf@counta by1%
\fi}%
\edef\pgfmathresult{\the\c@pgf@counta}%
\pgfmathsmuggle\pgfmathresult\endgroup%
}
\pgfmathdeclarefunction*{x2d}{1}{%
\begingroup%
\pgf@process{\pgfpointanchor{#1}{center}}%
\pgfmathparse{\pgf@x}%
\pgfmathsmuggle\pgfmathresult\endgroup%
}%
\pgfmathdeclarefunction*{y2d}{1}{%
\begingroup%
\pgf@process{\pgfpointanchor{#1}{center}}%
\pgfmathparse{\pgf@y}%
\pgfmathsmuggle\pgfmathresult\endgroup%
}%
%
\tikzset{convex hull of/.code={% expects a list of points
\edef\pgfutil@tmpl{#1}% store the list for manipulations
\edef\pgfutil@tmpp{}% path to be constructed
\c@pgf@counta1%
\pgfutil@for\pgfutil@tmpa:={\pgfutil@tmpl}\do{%
\ifnum\c@pgf@counta=1\relax
\pgfmathsetmacro{\pgfutil@tmpb}{x2d("\pgfutil@tmpa")}% min
\edef\pgfutil@tmpd{\pgfutil@tmpa}% leftmost point
\else
\pgfmathsetmacro{\pgfutil@tmpc}{x2d("\pgfutil@tmpa")}% x coordinate
\ifdim\pgfutil@tmpc pt<\pgfutil@tmpb pt\relax
\edef\pgfutil@tmpb{\pgfutil@tmpc}% new min
\edef\pgfutil@tmpd{\pgfutil@tmpa}% new leftmost point
\fi
\fi
\advance\c@pgf@counta by1%
}
\edef\pgfutil@tmpp{(\pgfutil@tmpd)}%
\edef\pgfutil@tmpj{\pgfutil@tmpd}%
\edef\pgfutil@tmpe{90}% slope correction
\edef\pgfutil@tmpk{1}% first round flag
\loop
%\typeout{Removing  \pgfutil@tmpd}%
\edef\pgfutil@tmph{\noexpand\@removeelement{\pgfutil@tmpd}{\pgfutil@tmpl}{\noexpand\pgfutil@tmpl}}%
\pgfutil@tmph
%\typeout{List is now \pgfutil@tmpl}%
\c@pgf@countb1%
\pgfutil@for\pgfutil@tmpa:={\pgfutil@tmpl}\do{%
\pgfmathsetmacro{\pgfutil@tmpf}{Mod(atan2(y2d("\pgfutil@tmpa")-y2d("\pgfutil@tmpd"),
x2d("\pgfutil@tmpa")-x2d("\pgfutil@tmpd"))+\pgfutil@tmpe,360)}%
%\typeout{\the\c@pgf@countb: \pgfutil@tmpa\space has angle \pgfutil@tmpf.}%
\ifnum\c@pgf@countb=1\relax
\edef\pgfutil@tmpg{\pgfutil@tmpf}%
\edef\pgfutil@tmpi{\pgfutil@tmpa}%
\else
%\unless\ifdim\pgfutil@tmpf pt>180pt\relax
\ifdim\pgfutil@tmpf pt>\pgfutil@tmpg pt\relax
\edef\pgfutil@tmpg{\pgfutil@tmpf}%
\edef\pgfutil@tmpi{\pgfutil@tmpa}%
\fi
%\fi
\fi
\advance\c@pgf@countb by1%
}
%
\ifx\pgfutil@tmpi\pgfutil@tmpj\relax
\c@pgf@counta=2\relax
\else
\edef\pgfutil@tmpd{\pgfutil@tmpi}%
\edef\pgfutil@tmpp{\pgfutil@tmpp--(\pgfutil@tmpd)}%
\pgfmathsetmacro{\pgfutil@tmpe}{180+\pgfutil@tmpe-\pgfutil@tmpf}%
%\typeout{offset angle=\pgfutil@tmpe}%
\advance\c@pgf@counta by-1%
\fi
\ifnum\pgfutil@tmpk=1\relax
\edef\pgfutil@tmpl{\pgfutil@tmpl,\pgfutil@tmpj}% add the first point again
\edef\pgfutil@tmpk{0}%
\fi
\ifnum\c@pgf@counta>2\relax
\repeat
\edef\pgfutil@tmpp{\pgfutil@tmpp--cycle}%
%\typeout{hull path=\pgfutil@tmpp}%
\tikzset{insert path=\pgfutil@tmpp}%
}}
%
\tikzset{generous outside path/.style={overlay,insert path={[reset cm]
			(-16383.99999pt,-16383.99999pt) |-
			(16383.99999pt,16383.99999pt)|- cycle}},
		 generous outside path'/.style={overlay,insert path={[reset cm]
			(-16383.99999pt,-16383.99999pt) -|
			(16383.99999pt,16383.99999pt)-| cycle}}}
%			
\tikzset{even odd clip/.code={\pgfseteorule}}			
%
% the following is very much "inspired" by the calc library 
\long\def\pgfmathtdparse#1{% < and > really are placeholders for a later integration
\begingroup% into calc, which however requires changes both in
% tikzlibrarycalc.code.tex and in tikz.code.tex
% 
% tdparse main computation. It's a series of optional factors in front 
% of coordinates. It is very much copied from the calc library. 
% maybe one can implement groups via [ ... ]
% 
\edef\pgf@Xa{0pt}% We accumulate the result in here. 
\edef\pgf@Ya{0pt}% 
\edef\pgf@Za{0pt}% 
\tikz@td@cc@parse+#1% 
}% 

\def\tikz@td@cc@parse{% 
\pgfutil@ifnextchar>{% 
% Ok, we found the end... 
\tikz@td@cc@end% 
} 
{\pgfutil@ifnextchar+{% 
% Ok, we found a coordinate... 
\tikz@td@cc@add% 
}{% 
\pgfutil@ifnextchar-{% 
\tikz@td@cc@sub% 
}{% 
\pgfutil@ifnextchar x{% 
\tikz@td@cc@vecprod% 
}{% 
\pgfutil@ifnextchar o{% 
\tikz@td@cc@scalprod% 
}{% \tikzerror{+ or - expected}% 
\tikz@td@cc@end%
}% 
}% 
}% 
}% 
}%
}%
% 
% The end is reached with > at the moment but this should change 
% 
\def\tikz@td@cc@end{% 
\ifcase\tikz@td@type%
\pgfmathsetmacro{\pgftemp@x}{\pgf@Xa}% 
\pgfmathsetmacro{\pgftemp@y}{\pgf@Ya}% 
\pgfmathsetmacro{\pgftemp@z}{\pgf@Za}% 
\edef\pgfmathresult{\pgftemp@x,\pgftemp@y,\pgftemp@z}%
\or%
\pgfmathsetmacro{\myxa}{\pgf@Xa}%
\pgfmathsetmacro{\myya}{\pgf@Ya}%
\pgfmathsetmacro{\myza}{\pgf@Za}%
\pgfmathsetmacro{\myxb}{\pgf@Xb}%
\pgfmathsetmacro{\myyb}{\pgf@Yb}%
\pgfmathsetmacro{\myzb}{\pgf@Zb}%
\pgfmathsetmacro{\pgftemp@x}{\vpauxx{\myxb}{\myyb}{\myzb}{\myxa}{\myya}{\myza}}% 
\pgfmathsetmacro{\pgftemp@y}{\vpauxy{\myxb}{\myyb}{\myzb}{\myxa}{\myya}{\myza}}% 
\pgfmathsetmacro{\pgftemp@z}{\vpauxz{\myxb}{\myyb}{\myzb}{\myxa}{\myya}{\myza}}% 
%\typeout{P1=(\myxb,\myyb,\myzb),P2=(\myxa,\myya,\myza),P1xP2=(\pgftemp@x,\pgftemp@y,\pgftemp@z)}%
\edef\pgfmathresult{\pgftemp@x,\pgftemp@y,\pgftemp@z}%
\or%
\pgfmathsetmacro{\myxa}{\pgf@Xa}%
\pgfmathsetmacro{\myya}{\pgf@Ya}%
\pgfmathsetmacro{\myza}{\pgf@Za}%
\pgfmathsetmacro{\myxb}{\pgf@Xb}%
\pgfmathsetmacro{\myyb}{\pgf@Yb}%
\pgfmathsetmacro{\myzb}{\pgf@Zb}%
\pgfmathparse@td@FPU{\myxa*\myxb+\myya*\myyb+\myza*\myzb}%
%\typeout{P1=(\myxb,\myyb,\myzb),P2=(\myxa,\myya,\myza),P1.P2=(\pgmfmathresult)}%
\fi%
%\message{result = (\pgftemp@x,\pgftemp@y,\pgftemp@z)=\pgfmathresult^^J}%
\xdef\tikz@td@type{0}% reset type to linear combination
\pgfmathsmuggle\pgfmathresult\endgroup}% 
% 
\def\tikz@td@cc@add+{% 
\def\tikz@td@cc@factor{1}% 
\tikz@td@cc@factororcoordinate% 
}% 
\def\tikz@td@cc@sub-{% 
\def\tikz@td@cc@factor{-1}% 
\tikz@td@cc@factororcoordinate% 
}% 
\def\tikz@td@cc@vecprod x{% 
%\message{Ah, a vector product^^J}%
\xdef\tikz@td@type{1}%
\edef\pgf@Xb{\pgf@Xa}% store current vector in b
\edef\pgf@Yb{\pgf@Ya}%
\edef\pgf@Zb{\pgf@Za}%
\edef\pgf@Xa{0pt}% reset a
\edef\pgf@Ya{0pt}%
\edef\pgf@Za{0pt}%
\def\tikz@td@cc@factor{1}%
\tikz@td@cc@factororcoordinate% 
}%
\def\tikz@td@cc@scalprod o{% 
%\message{Ah, a scalar product^^J}%
\xdef\tikz@td@type{2}%
\edef\pgf@Xb{\pgf@Xa}% store current vector in b
\edef\pgf@Yb{\pgf@Ya}%
\edef\pgf@Zb{\pgf@Za}%
\edef\pgf@Xa{0pt}% reset a
\edef\pgf@Ya{0pt}%
\edef\pgf@Za{0pt}%
\def\tikz@td@cc@factor{1}%
\tikz@td@cc@factororcoordinate% 
}%
% 
% Check for a factor: If we see a (, its a coordinate... 
% 
\def\tikz@td@cc@factororcoordinate{% 
\pgfutil@ifnextchar({%) 
% Ok, found coordinate 
\tikz@td@cc@coordinate% 
}{% 
\tikz@td@cc@parse@factor% 
}% 
}% 
% 
% ... otherwise it's a factor. It ends at ...*( 
% 
\def\tikz@td@cc@parse@factor#1*({% 
\pgfmathparse@td@FPU{#1*\tikz@td@cc@factor}% 
\let\tikz@td@cc@factor=\pgfmathresult% 
\tikz@td@cc@coordinate(%) 
}% 
\def\tikz@td@cc@coordinate(#1){% 
\ParseCoord(#1)% 
\pgfmathsetmacro{\pgf@Xa}{\pgf@Xa+\tikz@td@cc@factor*\pgf@X}%
\pgfmathsetmacro{\pgf@Ya}{\pgf@Ya+\tikz@td@cc@factor*\pgf@Y}%
\pgfmathsetmacro{\pgf@Za}{\pgf@Za+\tikz@td@cc@factor*\pgf@Z}%
\tikz@td@cc@parse% 
}% 
\tikzset{declare function={torusx(\u,\v,\R,\r)=cos(\u)*(\R + \r*cos(\v)); 
torusy(\u,\v,\R,\r)=(\R + \r*cos(\v))*sin(\u);
torusz(\u,\v,\R,\r)=\r*sin(\v);
torusvcrit1(\u,\th)=atan(tan(\th)*sin(\u));% first critical v value
torusvcrit2(\u,\th)=180+atan(tan(\th)*sin(\u));% second critical v value
torusvtest(\u,\v,\az,\el)=sin(-vcrit1(\u-\az,\el)+\v);
torusdisc(\th,\R,\r)=((pow(\r,2)-pow(\R,2))*pow(cot(\th),2)+% 
pow(\r,2)*(2+pow(tan(\th),2)))/pow(\R,2);% discriminant
torusumax(\th,\R,\r)=ifthenelse(torusdisc(\th,\R,\r)>0,asin(sqrt(abs(torusdisc(\th,\R,\r)))),0);
}}%
%
\tikzset{3d parse/.style={/utils/exec=\pgfmathtdparse{#1},%
insert path={(\pgfmathresult)}},3d coordinate/.style args={#1=#2}{%
/utils/exec=\pgfmathtdparse{#2},%
insert path={(\pgfmathresult) coordinate #1}}}%
\newcommand\pgfmathprintvector[2][]{%
\begingroup
\pgfkeys{#1}%
\pgfutil@tempcnta0%
\pgfutil@for\pgf@tmp:={#2}\do{\advance\pgfutil@tempcnta by1}%
\pgfutil@tempcntb1%
\pgfutil@for\pgf@tmp:={#2}\do{\advance\pgfutil@tempcntb by1%
\ifnum\pgfutil@tempcntb<\pgfutil@tempcnta
\pgfmathparse{\pgf@tmp}%
\pgfmathprintnumber{\pgfmathresult},%
\else
\pgfmathparse{\pgf@tmp}%
\pgfmathprintnumber{\pgfmathresult}%
\fi}%
\endgroup
}%
% 
% projection of a point on a line
\tikzset{3d/projection precomputations/.code n args={3}{%
\pgfmathsetmacro{\pgfutil@tmpa}{TD("#2-#3o#1-#3")/TD("#2-#3o#2-#3")}%
\pgfmathsetmacro{\pgfutil@tmpb}{TD("#3+\pgfutil@tmpa*#2-\pgfutil@tmpa*#3")}%
},3d/projection of/.style args={#1 on #2--#3}{%
3d/projection precomputations={#1}{#2}{#3},%
insert path={[3d parse={(\pgfutil@tmpb)}]}
}}
%
% define extended objects and project point on line or plane
\tikzset{3d/.cd,
plane through/.code args={#1 and #2 and #3 named #4}{%
%\typeout{#2-#1x#3-#1}%
\pgfmathsetmacro{\pgfutil@tmpa}{TD("#2-#1x#3-#1")}%
\pgfmathsetmacro{\pgfutil@tmpb}{TD("(\pgfutil@tmpa)o(\pgfutil@tmpa)")}%
\ifdim\pgfutil@tmpb pt<0.01pt\relax
\PackageWarning{3dtools}{The points #1 and #2 and #3 are either too close to each other or almost
on one line, so not suited to define a plane.}%
\else
\pgfmathsetmacro{\pgfutil@tmpc}{1/sqrt(\pgfutil@tmpb)}%
\pgfmathsetmacro{\pgfutil@tmpd}{TD("\pgfutil@tmpc*(\pgfutil@tmpa)")}%
\pgfmathsetmacro{\pgfutil@tmpe}{TD("#1")}%
%\typeout{normal=(\pgfutil@tmpd),point=(\pgfutil@tmpe),name=#4}%
\expandafter\xdef\csname tikz@td@plane@normal@#4\endcsname{\pgfutil@tmpd}%
\expandafter\xdef\csname tikz@td@plane@point@#4\endcsname{\pgfutil@tmpe}%
\expandafter\xdef\csname tikz@td@dim@#4\endcsname{2}%
\fi
},%
plane with normal/.code args={#1 through #2 named #3}{%
\pgfmathsetmacro{\pgfutil@tmpd}{TD("#1")}%
\pgfmathsetmacro{\pgfutil@tmpe}{TD("#2")}%
\expandafter\xdef\csname tikz@td@plane@normal@#3\endcsname{\pgfutil@tmpd}%
\expandafter\xdef\csname tikz@td@plane@point@#3\endcsname{\pgfutil@tmpe}%
\expandafter\xdef\csname tikz@td@dim@#3\endcsname{2}%
},%
line through/.code args={#1 and #2 named #3}{%
\pgfmathsetmacro{\pgfutil@tmpd}{TD("#1")}%
\pgfmathsetmacro{\pgfutil@tmpe}{TD("#2")}%
\pgfmathsetmacro{\pgfutil@tmpf}{TD("#1-#2o#1-#2")}%
\ifdim\pgfutil@tmpf pt<0.001pt\relax
\PackageWarning{3dtools}{The points #1 and #2 are too close to each other. No line defined.}%
\else
\expandafter\xdef\csname tikz@td@line@A@#3\endcsname{\pgfutil@tmpd}%
\expandafter\xdef\csname tikz@td@line@B@#3\endcsname{\pgfutil@tmpe}%
\expandafter\xdef\csname tikz@td@dim@#3\endcsname{1}%
\fi
},%
line with direction/.code args={#1 through #2 named #3}{%
\pgfmathsetmacro{\pgfutil@tmpd}{TD("#2")}%
\pgfmathsetmacro{\pgfutil@tmpe}{TD("#1+#2")}%
\pgfmathsetmacro{\pgfutil@tmpf}{TD("#1o#1")}%
\ifdim\pgfutil@tmpf pt<0.001pt\relax
\PackageWarning{3dtools}{The points #1 and #2 are too close to each other. No line defined.}%
\else
\expandafter\xdef\csname tikz@td@line@A@#3\endcsname{\pgfutil@tmpd}%
\expandafter\xdef\csname tikz@td@line@B@#3\endcsname{\pgfutil@tmpe}%
\expandafter\xdef\csname tikz@td@dim@#3\endcsname{1}%
\fi
},%
sphere with center/.code args={#1 and radius #2 named #3}{%
\pgfmathsetmacro{\pgfutil@tmpd}{TD("#1")}%
\expandafter\xdef\csname tikz@td@sphere@C@#3\endcsname{\pgfutil@tmpd}%
\expandafter\xdef\csname tikz@td@sphere@r@#3\endcsname{#2}%
\expandafter\xdef\csname tikz@td@dim@#3\endcsname{3}% reserve 3 for sphere...
},
project/.code args={#1 on #2}{%
\expandafter\edef\expandafter\pgfutil@tmpd{\csname tikz@td@dim@#2\endcsname}%
\ifcase\pgfutil@tmpd
\PackageWarning{3dtools}{It is not particularly meaningful to project a point on a point.}
\or% line
\expandafter\edef\expandafter\pgfutil@tmpa{\csname tikz@td@line@A@#2\endcsname}%
\expandafter\edef\expandafter\pgfutil@tmpb{\csname tikz@td@line@B@#2\endcsname}%
\pgfmathsetmacro{\pgfutil@tmpd}{TD("(\pgfutil@tmpa)-(\pgfutil@tmpb)o#1-(\pgfutil@tmpb)")/TD("(\pgfutil@tmpa)-(\pgfutil@tmpb)o(\pgfutil@tmpa)-(\pgfutil@tmpb)")}%
\pgfmathsetmacro{\pgfutil@tmpc}{TD("(\pgfutil@tmpb)+\pgfutil@tmpd*(\pgfutil@tmpa)-\pgfutil@tmpd*(\pgfutil@tmpb)")}%
%\typeout{alpha=\pgfutil@tmpd,projection=(\pgfutil@tmpc)}%
\tikzset{insert path={(\pgfutil@tmpc)}}%
\or% plane
\expandafter\edef\expandafter\pgfutil@tmpn{\csname tikz@td@plane@normal@#2\endcsname}%
\expandafter\edef\expandafter\pgfutil@tmpa{\csname tikz@td@plane@point@#2\endcsname}%
\pgfmathsetmacro{\pgfutil@tmpb}{TD("(\pgfutil@tmpn)o(\pgfutil@tmpa)-#1")}%
\pgfmathsetmacro{\pgfutil@tmpc}{TD("#1+\pgfutil@tmpb*(\pgfutil@tmpn)")}%
%\typeout{alpha=\pgfutil@tmpb,projection=(\pgfutil@tmpc)}%
\tikzset{insert path={(\pgfutil@tmpc)}}%
\fi
}}
%%
%% intersections in 3d (so far: intersection of line with line, plane, or sphere)
\tikzset{3d/intersection of/.code args={#1 with #2}{%
\expandafter\edef\expandafter\pgfutil@tmpd{\csname tikz@td@dim@#1\endcsname}%
\expandafter\edef\expandafter\pgfutil@tmpe{\csname tikz@td@dim@#2\endcsname}%
\edef\pgfutil@tmpde{\pgfutil@tmpd\pgfutil@tmpe}%
\edef\pgfutil@tmpf{0}% flag
\ifnum\pgfutil@tmpde=11\relax% line with line
 \expandafter\edef\expandafter\pgfutil@tmpa{\csname tikz@td@line@B@#1\endcsname}%
 \expandafter\edef\expandafter\pgfutil@tmpb{\csname tikz@td@line@B@#2\endcsname}%
 \expandafter\edef\expandafter\pgfutil@tmpc{\csname tikz@td@line@A@#1\endcsname}%
 \expandafter\edef\expandafter\pgfutil@tmpd{\csname tikz@td@line@A@#2\endcsname}%
\edef\pgfutil@tmpf{3}% 
\else
\ifnum\pgfutil@tmpde=21\relax% plane with line
 \expandafter\edef\expandafter\pgfutil@tmpa{\csname tikz@td@line@A@#2\endcsname}%
 \expandafter\edef\expandafter\pgfutil@tmpb{\csname tikz@td@line@B@#2\endcsname}%
 \expandafter\edef\expandafter\pgfutil@tmpn{\csname tikz@td@plane@normal@#1\endcsname}%
 \expandafter\edef\expandafter\pgfutil@tmpc{\csname tikz@td@plane@point@#1\endcsname}%
 \edef\pgfutil@tmpf{1}%
 \else
  \ifnum\pgfutil@tmpde=12\relax% line with plane
   \expandafter\edef\expandafter\pgfutil@tmpa{\csname tikz@td@line@A@#1\endcsname}%
   \expandafter\edef\expandafter\pgfutil@tmpb{\csname tikz@td@line@B@#1\endcsname}%
   \expandafter\edef\expandafter\pgfutil@tmpn{\csname tikz@td@plane@normal@#2\endcsname}%
   \expandafter\edef\expandafter\pgfutil@tmpc{\csname tikz@td@plane@point@#2\endcsname}%
   \edef\pgfutil@tmpf{1}% flag
  \else
   \ifnum\pgfutil@tmpde=13\relax% line with sphere
	\expandafter\edef\expandafter\pgfutil@tmpa{\csname tikz@td@line@A@#1\endcsname}%
	\expandafter\edef\expandafter\pgfutil@tmpb{\csname tikz@td@line@B@#1\endcsname}%
	\expandafter\edef\expandafter\pgfutil@tmpc{\csname tikz@td@sphere@C@#2\endcsname}%
	\expandafter\edef\expandafter\pgfutil@tmpr{\csname tikz@td@sphere@r@#2\endcsname}%
	\edef\pgfutil@tmpf{2}%
	\typeout{line with sphere}%
   \else
	\ifnum\pgfutil@tmpde=31\relax% sphere with line
	\expandafter\edef\expandafter\pgfutil@tmpa{\csname tikz@td@line@A@#2\endcsname}%
	\expandafter\edef\expandafter\pgfutil@tmpb{\csname tikz@td@line@B@#2\endcsname}%
	\expandafter\edef\expandafter\pgfutil@tmpc{\csname tikz@td@sphere@C@#1\endcsname}%
	\expandafter\edef\expandafter\pgfutil@tmpr{\csname tikz@td@sphere@r@#1\endcsname}%
	\edef\pgfutil@tmpf{2}%
    \typeout{sphere with line}%
	\else
	 \PackageWarning{3dtools}{Currently, one intersection object has to be a plane and the other 
  	   one a line. However, #1 has dimension \pgfutil@tmpd\space and 
	   #2 has dimension \pgfutil@tmpe.}%
	\fi
   \fi	
  \fi
 \fi
\fi
\ifnum\pgfutil@tmpf=1\relax% reasonable input, stored in a, b, c and n
%\typeout{debug: A=(\pgfutil@tmpa), B=(\pgfutil@tmpb), C=(\pgfutil@tmpc), n=(\pgfutil@tmpn)}%
\pgfmathsetmacro{\pgfutil@tmpg}{TD("(\pgfutil@tmpn)o(\pgfutil@tmpb)-(\pgfutil@tmpa)")}%
\pgfmathtruncatemacro{\pgfutil@tmph}{abs(\pgfutil@tmpg)>0.01?1:0}% 
\ifnum\pgfutil@tmph=1\relax%
\pgfmathsetmacro{\pgfutil@tmpi}{TD("(\pgfutil@tmpn)o(\pgfutil@tmpc)-(\pgfutil@tmpa)")/\pgfutil@tmpg}%
\pgfmathsetmacro{\pgfutil@tmpj}{TD("(\pgfutil@tmpa)+\pgfutil@tmpi*(\pgfutil@tmpb)-\pgfutil@tmpi*(\pgfutil@tmpa)")}%
\tikzset{insert path={(\pgfutil@tmpj)}}%
%\typeout{debug: intersection is at (\pgfutil@tmpj)}%
\else
\PackageWarning{3dtools}{The plane and line do not intersect within the uncertainties. Returning (0,0,0) instead.}%
\fi
\fi
\ifnum\pgfutil@tmpf=2\relax% line and sphere
\pgfmathsetmacro@td@FPU{\pgfutil@tmpg}{TD("(\pgfutil@tmpb)-(\pgfutil@tmpa)o(\pgfutil@tmpb)-(\pgfutil@tmpa)")}%
\pgfmathsetmacro@td@FPU{\pgfutil@tmph}{2*TD("(\pgfutil@tmpa)-(\pgfutil@tmpc)o(\pgfutil@tmpb)-(\pgfutil@tmpa)")}%
\pgfmathsetmacro@td@FPU{\pgfutil@tmpi}{TD("(\pgfutil@tmpa)-(\pgfutil@tmpc)o(\pgfutil@tmpa)-(\pgfutil@tmpc)")-\pgfutil@tmpr*\pgfutil@tmpr}%
\pgfmathsetmacro@td@FPU{\pgfutil@tmpj}{sign(\pgfutil@tmph*\pgfutil@tmph-4*\pgfutil@tmpg*\pgfutil@tmpi)}%
\ifdim\pgfutil@tmpj pt<0pt\relax
\PackageWarning{3dtools}{The sphere and line do not intersect within the uncertainties. Returning (0,0,0) instead.}%
\else
\pgfmathsetmacro@td@FPU{\pgfutil@tmpk}{-1*(\pgfutil@tmph+sqrt(\pgfutil@tmph*\pgfutil@tmph-4*\pgfutil@tmpg*\pgfutil@tmpi))/2/\pgfutil@tmpg}%
%\typeout{k=\pgfutil@tmpk}
\edef\pgfutil@tmpm{0}%
\ifdim\pgfutil@tmpk pt>0pt\relax
 \ifdim\pgfutil@tmpk pt<1pt\relax
  \edef\pgfutil@tmpm{1}%
  \pgfmathsetmacro{\pgfutil@tmpn}{TD("(\pgfutil@tmpa)+\pgfutil@tmpk*(\pgfutil@tmpb)-\pgfutil@tmpk*(\pgfutil@tmpa)")}%
  \tikzset{insert path={(\pgfutil@tmpn) coordinate (\pgfkeysvalueof{/tikz/3d/aux keys/intersection 1})}}%
  %\typeout{first intersection inserted.}%
 \fi
\fi
\pgfmathsetmacro@td@FPU{\pgfutil@tmpk}{-1*(\pgfutil@tmph-sqrt(\pgfutil@tmph*\pgfutil@tmph-4*\pgfutil@tmpg*\pgfutil@tmpi))/2/\pgfutil@tmpg}%
%\typeout{k=\pgfutil@tmpk}
\ifdim\pgfutil@tmpk pt>0pt\relax
 \ifdim\pgfutil@tmpk pt<1pt\relax
  \pgfmathsetmacro{\pgfutil@tmpn}{TD("(\pgfutil@tmpa)+\pgfutil@tmpk*(\pgfutil@tmpb)-\pgfutil@tmpk*(\pgfutil@tmpa)")}%
  \ifnum\pgfutil@tmpm=1
   \tikzset{insert path={(\pgfutil@tmpn) coordinate (\pgfkeysvalueof{/tikz/3d/aux keys/intersection 2})}}%
  \else
   \tikzset{insert path={(\pgfutil@tmpn) coordinate (\pgfkeysvalueof{/tikz/3d/aux keys/intersection 1})}}%
  \fi
 \fi
\fi
% cont
\fi
\fi
\ifnum\pgfutil@tmpf=3\relax% intersection between two lines, follow https://math.stackexchange.com/a/271366
 \pgfmathsetmacro{\pgfutil@tmpC}{TD("(\pgfutil@tmpc)")}%
 \pgfmathsetmacro{\pgfutil@tmpD}{TD("(\pgfutil@tmpd)")}%
 \pgfmathsetmacro{\pgfutil@tmpi}{sqrt(TD("(\pgfutil@tmpC)-(\pgfutil@tmpD)o(\pgfutil@tmpC)-(\pgfutil@tmpD)"))}%
 \ifdim\pgfutil@tmpi pt<0.05pt\relax
  \tikzset{insert path={(\pgfutil@tmpC) coordinate (\pgfkeysvalueof{/tikz/3d/aux keys/intersection 1})}}%
 \else
  \pgfmathsetmacro{\pgfutil@tmpE}{TD("(\pgfutil@tmpc)-(\pgfutil@tmpa)")}%
  \pgfmathsetmacro{\pgfutil@tmpF}{TD("(\pgfutil@tmpd)-(\pgfutil@tmpb)")}%
  \pgfmathsetmacro{\pgfutil@tmpG}{TD("(\pgfutil@tmpD)-(\pgfutil@tmpC)")}%
  \pgfmathsetmacro{\pgfutil@tmpH}{TD("(\pgfutil@tmpF)x(\pgfutil@tmpG)")}%
  \pgfmathsetmacro{\pgfutil@tmpK}{TD("(\pgfutil@tmpF)x(\pgfutil@tmpE)")}%
  \pgfmathsetmacro@td@FPU{\pgfutil@tmph}{sqrt(TD("(\pgfutil@tmpH)o(\pgfutil@tmpH)"))}%
  \pgfmathsetmacro@td@FPU{\pgfutil@tmpk}{sqrt(TD("(\pgfutil@tmpK)o(\pgfutil@tmpK)"))}%
  \ifdim\pgfutil@tmph pt<0.01pt\relax
   \PackageWarning{3dtools}{The lines do not intersect within the uncertainties. Returning (0,0,0) instead.}%
  \else
   \ifdim\pgfutil@tmpk pt<0.01pt\relax
	\PackageWarning{3dtools}{The lines do not intersect within the uncertainties. Returning (0,0,0) instead.}%
   \else
	\pgfmathsetmacro@td@FPU{\pgfutil@tmpl}{sign(TD("(\pgfutil@tmpH)o(\pgfutil@tmpK)"))*\pgfutil@tmph/\pgfutil@tmpk}%
	\pgfmathsetmacro{\pgfutil@tmpM}{TD("(\pgfutil@tmpC)+\pgfutil@tmpl*(\pgfutil@tmpE)")}%
	\tikzset{insert path={(\pgfutil@tmpM) coordinate (\pgfkeysvalueof{/tikz/3d/aux keys/intersection 1})}}%
   \fi 
  \fi
 \fi
\fi
}}
%%
%% circumsphere center
\tikzset{3d/circumsphere center/.code={%
\tikzset{3d/aux keys/.cd,#1}%
\edef\pgfutil@tmpA{\pgfkeysvalueof{/tikz/3d/aux keys/A}}%
\edef\pgfutil@tmpB{\pgfkeysvalueof{/tikz/3d/aux keys/B}}%
\edef\pgfutil@tmpC{\pgfkeysvalueof{/tikz/3d/aux keys/C}}%
\edef\pgfutil@tmpD{\pgfkeysvalueof{/tikz/3d/aux keys/D}}%
% shifting to A=0 by redefining B, C and D
\pgfmathsetmacro{\pgfutil@tmpB}{TD("\pgfutil@tmpB-\pgfutil@tmpA")}%
\pgfmathsetmacro{\pgfutil@tmpC}{TD("\pgfutil@tmpC-\pgfutil@tmpA")}%
\pgfmathsetmacro{\pgfutil@tmpD}{TD("\pgfutil@tmpD-\pgfutil@tmpA")}%
% 
\pgfmathsetmacro{\pgfutil@tmpBxC}{TD("(\pgfutil@tmpB)x(\pgfutil@tmpC)")}%
\pgfmathsetmacro{\pgfutil@tmpCxD}{TD("(\pgfutil@tmpC)x(\pgfutil@tmpD)")}%
\pgfmathsetmacro{\pgfutil@tmpDxB}{TD("(\pgfutil@tmpD)x(\pgfutil@tmpB)")}%
% determinant (times 2)
\pgfmathsetmacro{\pgfutil@tmpd}{2*TD("(\pgfutil@tmpB)o(\pgfutil@tmpCxD)")}%
%\typeout{Determinant is \pgfutil@tmpd.}
\pgfmathtruncatemacro{\pgfutil@tmpi}{abs(\pgfutil@tmpd)>0.02?1:0}%
\ifnum\pgfutil@tmpi=0\relax
\PackageWarning{3dtools}{The points approximately sit on a common plane. The circumsphere cannot
be determined.}%
\else
\pgfmathsetmacro{\pgfutil@tmpnB}{TD("(\pgfutil@tmpB)o(\pgfutil@tmpB)")/\pgfutil@tmpd}%
\pgfmathsetmacro{\pgfutil@tmpnC}{TD("(\pgfutil@tmpC)o(\pgfutil@tmpC)")/\pgfutil@tmpd}%
\pgfmathsetmacro{\pgfutil@tmpnD}{TD("(\pgfutil@tmpD)o(\pgfutil@tmpD)")/\pgfutil@tmpd}%
\pgfmathsetmacro{\pgfutil@tmpI}{TD("\pgfutil@tmpA+\pgfutil@tmpnB*(\pgfutil@tmpCxD)+\pgfutil@tmpnC*(\pgfutil@tmpDxB)+\pgfutil@tmpnD*(\pgfutil@tmpBxC)")}%
\tikzset{insert path={(\pgfutil@tmpI)}}%
\fi
},3d/circumcircle center/.code={%
\tikzset{3d/aux keys/.cd,#1}%
\edef\pgfutil@tmpA{\pgfkeysvalueof{/tikz/3d/aux keys/A}}%
\edef\pgfutil@tmpB{\pgfkeysvalueof{/tikz/3d/aux keys/B}}%
\edef\pgfutil@tmpC{\pgfkeysvalueof{/tikz/3d/aux keys/C}}%
% shifting to A=0 by redefining B and C
\pgfmathsetmacro{\pgfutil@tmpB}{TD("\pgfutil@tmpB-\pgfutil@tmpA")}%
\pgfmathsetmacro{\pgfutil@tmpC}{TD("\pgfutil@tmpC-\pgfutil@tmpA")}%
% 
\pgfmathsetmacro{\pgfutil@tmpBB}{TD("(\pgfutil@tmpB)o(\pgfutil@tmpB)")}%
\pgfmathsetmacro{\pgfutil@tmpBC}{TD("(\pgfutil@tmpB)o(\pgfutil@tmpC)")}%
\pgfmathsetmacro{\pgfutil@tmpCC}{TD("(\pgfutil@tmpC)o(\pgfutil@tmpC)")}%
% determinant (times 2)
\pgfmathsetmacro{\pgfutil@tmpd}{2*(\pgfutil@tmpBB*\pgfutil@tmpCC-\pgfutil@tmpBC*\pgfutil@tmpBC)}%
%\typeout{Determinant is \pgfutil@tmpd.}
\pgfmathtruncatemacro{\pgfutil@tmpi}{abs(\pgfutil@tmpd)>0.02?1:0}%
\ifnum\pgfutil@tmpi=0\relax
\PackageWarning{3dtools}{The points approximately sit on a common line. The circumcircle cannot
be determined.}%
\else
\pgfmathsetmacro{\pgfutil@tmpb}{(\pgfutil@tmpBB-\pgfutil@tmpBC)*\pgfutil@tmpCC/\pgfutil@tmpd}%
\pgfmathsetmacro{\pgfutil@tmpc}{(\pgfutil@tmpCC-\pgfutil@tmpBC)*\pgfutil@tmpBB/\pgfutil@tmpd}%
%\typeout{debug: b=\pgfutil@tmpb,c=\pgfutil@tmpc,(B)=(\pgfutil@tmpB),(C)=(\pgfutil@tmpC)}%
\pgfmathsetmacro{\pgfutil@tmpI}{TD("\pgfutil@tmpA+\pgfutil@tmpb*(\pgfutil@tmpB)+\pgfutil@tmpc*(\pgfutil@tmpC)")}%
%\typeout{debug: (I)=((\pgfutil@tmpI))}%
\tikzset{insert path={(\pgfutil@tmpI)}}%
\fi
},3d/define orthonormal dreibein/.code={%
\tikzset{3d/aux keys/.cd,#1}%
\edef\pgfutil@tmpA{\pgfkeysvalueof{/tikz/3d/aux keys/A}}%
\edef\pgfutil@tmpB{\pgfkeysvalueof{/tikz/3d/aux keys/B}}%
\edef\pgfutil@tmpC{\pgfkeysvalueof{/tikz/3d/aux keys/C}}%
\pgfmathsetmacro{\pgfutil@tmpB}{TD("\pgfutil@tmpB-\pgfutil@tmpA")}%
\pgfmathsetmacro{\pgfutil@tmpC}{TD("\pgfutil@tmpC-\pgfutil@tmpA")}%
\pgfmathsetmacro{\pgfutil@tmpD}{TD("(\pgfutil@tmpB)x(\pgfutil@tmpC)")}%
\pgfmathsetmacro{\pgfutil@tmpd}{TD("(\pgfutil@tmpD)o(\pgfutil@tmpD)")}%
\ifdim\pgfutil@tmpd pt<0.01pt\relax
\PackageWarning{3dtools}{The vectors \pgfutil@tmpA, \pgfutil@tmpB\space and 
\pgfutil@tmpC are too collinear. Cannot determine appropriate plane.}%
\else
\pgfmathsetmacro{\pgfutil@tmpb}{1/sqrt(TD("(\pgfutil@tmpB)o(\pgfutil@tmpB)"))}%
\pgfmathsetmacro{\pgfutil@tmpd}{1/sqrt(TD("(\pgfutil@tmpD)o(\pgfutil@tmpD)"))}%
\pgfmathsetmacro{\pgfutil@tmpB}{TD("\pgfutil@tmpb*(\pgfutil@tmpB)")}%
\pgfmathsetmacro{\pgfutil@tmpD}{TD("\pgfutil@tmpd*(\pgfutil@tmpD)")}%
\pgfmathsetmacro{\pgfutil@tmpC}{TD("(\pgfutil@tmpD)x(\pgfutil@tmpB)")}%
\edef\pgfutil@tmp{\noexpand\path[overlay,
	3d coordinate={\pgfkeysvalueof{/tikz/3d/aux keys/ex}=(\pgfutil@tmpB)},
	3d coordinate={\pgfkeysvalueof{/tikz/3d/aux keys/ey}=(\pgfutil@tmpC)},
	3d coordinate={\pgfkeysvalueof{/tikz/3d/aux keys/ez}=(\pgfutil@tmpD)}];}%
\pgfutil@tmp	
\fi
},3d/aux keys/.cd,A/.initial={(A)},B/.initial={(B)},
C/.initial={(C)},D/.initial={(D)},
intersection 1/.initial={I-1},intersection 2/.initial={I-2},
ex/.initial={(ex)},
ey/.initial={(ey)},ez/.initial={(ez)},
i1/.initial=i1,i2/.initial=i2,
rA/.initial=1,rB/.initial=1,rC/.initial=1}
%%
%% predefined pics
% based on https://en.wikipedia.org/wiki/Circumscribed_circle
\tikzset{pics/3d circle through 3 points/.style={code={%
 \tikzset{3d/circle through 3 points/.cd,#1}%
 \edef\temp{\noexpand\path[overlay,
3d coordinate={(\pgfkeysvalueof{/tikz/3d/circle through 3 points/auxiliary coordinate prefix}b)=\pgfkeysvalueof{/tikz/3d/circle through 3 points/A}-\pgfkeysvalueof{/tikz/3d/circle through 3 points/C}},
 3d coordinate={(\pgfkeysvalueof{/tikz/3d/circle through 3 points/auxiliary coordinate prefix}a)=\pgfkeysvalueof{/tikz/3d/circle through 3 points/B}-\pgfkeysvalueof{/tikz/3d/circle through 3 points/C}},
 3d coordinate={(\pgfkeysvalueof{/tikz/3d/circle through 3 points/auxiliary coordinate prefix}c)=\pgfkeysvalueof{/tikz/3d/circle through 3 points/A}-\pgfkeysvalueof{/tikz/3d/circle through 3 points/B}},
 3d coordinate={(\pgfkeysvalueof{/tikz/3d/circle through 3 points/auxiliary coordinate prefix}n)=(\pgfkeysvalueof{/tikz/3d/circle through 3 points/auxiliary coordinate prefix}a)x(\pgfkeysvalueof{/tikz/3d/circle through 3 points/auxiliary coordinate prefix}b)}
 ];}%
 \temp    
 \pgfmathsetmacro{\lengthn}{sqrt(TD("(\pgfkeysvalueof{/tikz/3d/circle through 3 points/auxiliary coordinate prefix}n)o(\pgfkeysvalueof{/tikz/3d/circle through 3 points/auxiliary coordinate prefix}n)"))}
 \ifdim\lengthn pt<0.02pt
   \message{The points are (almost) on a line. Circle cannot be determined.}
  \else
   \pgfmathsetmacro{\tmpradius}{sqrt(TD("(\pgfkeysvalueof{/tikz/3d/circle through 3 points/auxiliary coordinate prefix}a)o(\pgfkeysvalueof{/tikz/3d/circle through 3 points/auxiliary coordinate prefix}a)"))*%
   	sqrt(TD("(\pgfkeysvalueof{/tikz/3d/circle through 3 points/auxiliary coordinate prefix}b)o(\pgfkeysvalueof{/tikz/3d/circle through 3 points/auxiliary coordinate prefix}b)"))*sqrt(TD("(\pgfkeysvalueof{/tikz/3d/circle through 3 points/auxiliary coordinate prefix}c)o(\pgfkeysvalueof{/tikz/3d/circle through 3 points/auxiliary coordinate prefix}c)"))/%
		(2*\lengthn)}
   \pgfmathsetmacro{\coeffa}{-1*TD("(\pgfkeysvalueof{/tikz/3d/circle through 3 points/auxiliary coordinate prefix}b)o(\pgfkeysvalueof{/tikz/3d/circle through 3 points/auxiliary coordinate prefix}b)")/(2*TD("(\pgfkeysvalueof{/tikz/3d/circle through 3 points/auxiliary coordinate prefix}n)o(\pgfkeysvalueof{/tikz/3d/circle through 3 points/auxiliary coordinate prefix}n)"))}		
   \pgfmathsetmacro{\coeffb}{TD("(\pgfkeysvalueof{/tikz/3d/circle through 3 points/auxiliary coordinate prefix}a)o(\pgfkeysvalueof{/tikz/3d/circle through 3 points/auxiliary coordinate prefix}a)")/(2*TD("(\pgfkeysvalueof{/tikz/3d/circle through 3 points/auxiliary coordinate prefix}n)o(\pgfkeysvalueof{/tikz/3d/circle through 3 points/auxiliary coordinate prefix}n)"))}
   \edef\temp{%
   \noexpand\path[overlay,3d coordinate={(\pgfkeysvalueof{/tikz/3d/circle through 3 points/auxiliary coordinate prefix}u)=\coeffa*(\pgfkeysvalueof{/tikz/3d/circle through 3 points/auxiliary coordinate prefix}a)x(\pgfkeysvalueof{/tikz/3d/circle through 3 points/auxiliary coordinate prefix}n)},
     3d coordinate={(\pgfkeysvalueof{/tikz/3d/circle through 3 points/auxiliary coordinate prefix}v)=\coeffb*(\pgfkeysvalueof{/tikz/3d/circle through 3 points/auxiliary coordinate prefix}b)x(\pgfkeysvalueof{/tikz/3d/circle through 3 points/auxiliary coordinate prefix}n)}];
   \noexpand\path[3d coordinate={(\pgfkeysvalueof{/tikz/3d/circle through 3 points/center name})=\pgfkeysvalueof{/tikz/3d/circle through 3 points/C}+(\pgfkeysvalueof{/tikz/3d/circle through 3 points/auxiliary coordinate prefix}u)+(\pgfkeysvalueof{/tikz/3d/circle through 3 points/auxiliary coordinate prefix}v)}];
   }%
   \temp
   \pgfmathsetmacro{\normalizationa}{1/sqrt(TD("(\pgfkeysvalueof{/tikz/3d/circle through 3 points/auxiliary coordinate prefix}a)o(\pgfkeysvalueof{/tikz/3d/circle through 3 points/auxiliary coordinate prefix}a)"))}
   \pgfmathsetmacro{\normalizationn}{1/sqrt(TD("(\pgfkeysvalueof{/tikz/3d/circle through 3 points/auxiliary coordinate prefix}n)o(\pgfkeysvalueof{/tikz/3d/circle through 3 points/auxiliary coordinate prefix}n)"))}
   \edef\temp{%
   \noexpand\path[overlay,3d coordinate={(\pgfkeysvalueof{/tikz/3d/circle through 3 points/auxiliary coordinate prefix}a)=\normalizationa*(\pgfkeysvalueof{/tikz/3d/circle through 3 points/auxiliary coordinate prefix}a)},
   3d coordinate={(\pgfkeysvalueof{/tikz/3d/circle through 3 points/auxiliary coordinate prefix}n)=\normalizationn*(\pgfkeysvalueof{/tikz/3d/circle through 3 points/auxiliary coordinate prefix}n)},
   3d coordinate={(\pgfkeysvalueof{/tikz/3d/circle through 3 points/auxiliary coordinate prefix}c)=(\pgfkeysvalueof{/tikz/3d/circle through 3 points/auxiliary coordinate prefix}a)x(\pgfkeysvalueof{/tikz/3d/circle through 3 points/auxiliary coordinate prefix}n)}];
   }%
   \temp
   \edef\temp{%
	\noexpand\begin{scope}[plane x={(\pgfkeysvalueof{/tikz/3d/circle through 3 points/auxiliary coordinate prefix}a)},plane y={(\pgfkeysvalueof{/tikz/3d/circle through 3 points/auxiliary coordinate prefix}c)},canvas is plane]
	 \noexpand\draw[pic actions] (\pgfkeysvalueof{/tikz/3d/circle through 3 points/center name}) circle[radius=\tmpradius];
	 \noexpand\pgfkeys{/tikz/3d/circle through 3 points/extra}
	\noexpand\end{scope}}%
   \temp	
  \fi 
}},3d/circle through 3 points/.cd,A/.initial={(1,0,0)},B/.initial={(0,1,0)},
C/.initial={(0,0,1)},
auxiliary coordinate prefix/.initial=tmp,center name/.initial=M,
extra/.code={}}%
% 
% incircle
\tikzset{pics/3d incircle/.style={code={%
\tikzset{3d/incircle/.cd,#1}%
\pgfmathsetmacro{\pgfutil@tmpa}{tddistance("\pgfkeysvalueof{/tikz/3d/incircle/B}","\pgfkeysvalueof{/tikz/3d/incircle/C}")}%
\pgfmathsetmacro{\pgfutil@tmpb}{tddistance("\pgfkeysvalueof{/tikz/3d/incircle/A}","\pgfkeysvalueof{/tikz/3d/incircle/C}")}%
\pgfmathsetmacro{\pgfutil@tmpc}{tddistance("\pgfkeysvalueof{/tikz/3d/incircle/A}","\pgfkeysvalueof{/tikz/3d/incircle/B}")}%
\pgfmathsetmacro{\pgfutil@tmpd}{(\pgfutil@tmpa+\pgfutil@tmpb+\pgfutil@tmpc)/2}%
\pgfmathsetmacro{\pgfutil@tmparea}{sqrt(\pgfutil@tmpd*(\pgfutil@tmpd-\pgfutil@tmpa)*(\pgfutil@tmpd-\pgfutil@tmpb)*(\pgfutil@tmpd-\pgfutil@tmpc))}%
\pgfmathsetmacro{\pgfutil@tmpr}{\pgfutil@tmparea/\pgfutil@tmpd}%
\edef\pgfutil@tmpt{\noexpand\ParseCoord\pgfkeysvalueof{/tikz/3d/incircle/A}}%
\pgfutil@tmpt
\pgfmathsetmacro{\pgfutil@tmp@Ax}{\pgf@X}%
\pgfmathsetmacro{\pgfutil@tmp@Ay}{\pgf@Y}%
\pgfmathsetmacro{\pgfutil@tmp@Az}{\pgf@Z}%
\edef\pgfutil@tmpt{\noexpand\ParseCoord\pgfkeysvalueof{/tikz/3d/incircle/B}}%
\pgfutil@tmpt
\pgfmathsetmacro{\pgfutil@tmp@Bx}{\pgf@X}%
\pgfmathsetmacro{\pgfutil@tmp@By}{\pgf@Y}%
\pgfmathsetmacro{\pgfutil@tmp@Bz}{\pgf@Z}%
\edef\pgfutil@tmpt{\noexpand\ParseCoord\pgfkeysvalueof{/tikz/3d/incircle/C}}%
\pgfutil@tmpt
\pgfmathsetmacro{\pgfutil@tmp@Cx}{\pgf@X}%
\pgfmathsetmacro{\pgfutil@tmp@Cy}{\pgf@Y}%
\pgfmathsetmacro{\pgfutil@tmp@Cz}{\pgf@Z}%
\pgfmathsetmacro{\pgfutil@tmpIx}{\pgfutil@tmp@Ax*\pgfutil@tmpa+\pgfutil@tmp@Bx*\pgfutil@tmpb+\pgfutil@tmp@Cx*\pgfutil@tmpc)/\pgfutil@tmpd/2}%
\pgfmathsetmacro{\pgfutil@tmpIy}{\pgfutil@tmp@Ay*\pgfutil@tmpa+\pgfutil@tmp@By*\pgfutil@tmpb+\pgfutil@tmp@Cy*\pgfutil@tmpc)/\pgfutil@tmpd/2}%
\pgfmathsetmacro{\pgfutil@tmpIz}{\pgfutil@tmp@Az*\pgfutil@tmpa+\pgfutil@tmp@Bz*\pgfutil@tmpb+\pgfutil@tmp@Cz*\pgfutil@tmpc)/\pgfutil@tmpd/2}%
\path (\pgfutil@tmpIx,\pgfutil@tmpIy,\pgfutil@tmpIz) 
  coordinate (\pgfkeysvalueof{/tikz/3d/incircle/center name})
  [3d/projection of={(\pgfkeysvalueof{/tikz/3d/incircle/center name}) on 
	\pgfkeysvalueof{/tikz/3d/incircle/A}--\pgfkeysvalueof{/tikz/3d/incircle/B}}] 
 coordinate (\pgfkeysvalueof{/tikz/3d/incircle/projection prefix}c)
 [3d/projection of={(\pgfkeysvalueof{/tikz/3d/incircle/center name}) on 
	\pgfkeysvalueof{/tikz/3d/incircle/A}--\pgfkeysvalueof{/tikz/3d/incircle/C}}] 
 coordinate (\pgfkeysvalueof{/tikz/3d/incircle/projection prefix}b)
 [3d/projection of={(\pgfkeysvalueof{/tikz/3d/incircle/center name}) on 
	\pgfkeysvalueof{/tikz/3d/incircle/B}--\pgfkeysvalueof{/tikz/3d/incircle/C}}] 
 coordinate (\pgfkeysvalueof{/tikz/3d/incircle/projection prefix}a);
 \path[pic actions] pic{3d circle through 3 points={%
   	A={(\pgfkeysvalueof{/tikz/3d/incircle/projection prefix}c)},%
   	B={(\pgfkeysvalueof{/tikz/3d/incircle/projection prefix}b)},%
  	C={(\pgfkeysvalueof{/tikz/3d/incircle/projection prefix}a)},%
  	center name=\pgfkeysvalueof{/tikz/3d/incircle/center name}}}
	;
}},3d/incircle/.cd,
center name/.initial=I,
A/.initial={(1,0,0)},B/.initial={(0,1,0)},C/.initial={(0,0,1)},
projection prefix/.initial=tmpp}
%
\tikzset{pics/3d/cone/.style={code={%
	\tikzset{3d/.cd,#1}%
	\def\pv##1{\pgfkeysvalueof{/tikz/3d/##1}}%
	\pgfmathsetmacro{\sdtip}{screendepth(0,0,\pv{h})}
	\pgfmathsetmacro{\aspectangle}{atan2(\sdtip,sqrt(\pv{h}*\pv{h}-\sdtip*\sdtip))}
	\path (0,0,\pv{h}) coordinate (-tip);
    \begin{scope}[x={(0,0,tan(\aspectangle))},y={($(0,0,0)!1cm!90:(0,0,\pv{h})$)}]
	 \pgfmathtruncatemacro{\itest}{abs(tan(\aspectangle)*\pv{r}/\pv{h})<1}
	 \ifnum\itest=1
	  \pgfmathsetmacro{\alphacrit}{acos(tan(\aspectangle)*\pv{r}/\pv{h})}
	  \ifdim\sdtip pt>0pt
	   \path[/tikz/3d/hidden] (\alphacrit:\pv{r}) arc[start angle=\alphacrit,end
	   angle=-\alphacrit,radius=\pv{r}];
	   \path[/tikz/3d/visible] (\alphacrit:\pv{r}) 
	   arc[start angle=\alphacrit,end  angle=360-\alphacrit,radius=\pv{r}]
	   -- (-tip) -- cycle;
	  \else
	   \path[/tikz/3d/visible] circle[radius=\pv{r}];
	   \path[/tikz/3d/visible] (\alphacrit:\pv{r}) 
	   -- (-tip) -- (360-\alphacrit:\pv{r});
	  \fi
	 \else
	  \path[/tikz/3d/visible] circle[radius=\pv{r}];	 
	 \fi
    \end{scope}
	}}}
%%
\tikzset{pics/3d/frustum/.style={code={%
  \tikzset{3d/.cd,#1}
  \def\pv##1{\pgfkeysvalueof{/tikz/3d/##1}}%
  \pgfmathtruncatemacro{\itest}{abs(\pv{R}-\pv{r})>0.05}
  \ifnum\itest=0
   \PackageWarning{3dtools}{Radii R=\pv{R} and r=\pv{r} are too close.}%  
  \else
   \pgfmathsetmacro{\myH}{(\pv{R}*\pv{h}/(\pv{R}-\pv{r}))}%
   \pgfmathsetmacro{\sdtip}{screendepth(0,0,\myH)}%
   \pgfmathsetmacro{\aspectangle}{atan2(\sdtip,sqrt(\myH*\myH-\sdtip*\sdtip))}%
   %\typeout{H=\myH,angle=\aspectangle}
   \path[overlay] (0,0,\myH) coordinate (-tip)
   (0,0,\pv{h}) coordinate (-upper base)
   (0,0,{tan(\aspectangle)}) coordinate (-tdtmpex)
   ($(0,0,0)!1cm!90:(0,0,\pv{h})$) coordinate (-tdtmpey);
   \begin{scope}[x={(-tdtmpex)},y={(-tdtmpey)}]
	   \pgfmathtruncatemacro{\itest}{abs(tan(\aspectangle)*\pv{R}/\myH)<1}
	   \path[local bounding box=pft] (-upper base) circle[radius=\pv{r}]
	    (0,0,0) circle[radius=\pv{R}];
	   \ifnum\itest=1
		\pgfmathsetmacro{\alphacrit}{acos(tan(\aspectangle)*\pv{R}/\myH)}
		\ifdim\sdtip pt>0pt\relax
		 \begin{scope}
		  \clip (-upper base) circle[radius=\pv{r}] ;
		  \path[/tikz/3d/hidden] (\alphacrit:\pv{R}) arc[start angle=\alphacrit,end
		  angle=-\alphacrit,radius=\pv{R}];
		  \path[/tikz/3d/hidden] ($(-upper base)+(\alphacrit:\pv{r})$)
		  --(\alphacrit:\pv{R}) 
		  arc[start angle=\alphacrit,end  angle=360-\alphacrit,radius=\pv{R}]
		  -- ($(-upper base)+(360-\alphacrit:\pv{r})$);
		 \end{scope}
		 \begin{scope}
		  \clip[even odd clip] (-upper base) circle[radius=\pv{r}]  [generous outside path];
		  \path[/tikz/3d/hidden] (\alphacrit:\pv{R}) arc[start angle=\alphacrit,end
		  angle=-\alphacrit,radius=\pv{R}];
		  \path[/tikz/3d/visible] ($(-upper base)+(\alphacrit:\pv{r})$)
		  --(\alphacrit:\pv{R}) 
		  arc[start angle=\alphacrit,end  angle=360-\alphacrit,radius=\pv{R}]
		  -- ($(-upper base)+(360-\alphacrit:\pv{r})$);
		 \end{scope}
		 \path[/tikz/3d/visible] (-upper base) circle[radius=\pv{r}];
		\else %\typeout{\sdtip<0}
		 \begin{scope}
		  \clip circle[radius=\pv{R}];
		  \path[/tikz/3d/hidden] ($(-upper base)+(\alphacrit:\pv{r})$)
		  --(\alphacrit:\pv{R}) 
		  ($(-upper base)+(360-\alphacrit:\pv{r})$)
		  --(360-\alphacrit:\pv{R});
		  \path[/tikz/3d/hidden] (-upper base) circle[radius=\pv{r}];
		 \end{scope}
% 		 % enlarge the bounding box
% 		 \path ($(-upper base)+(\alphacrit:\pv{r})$)
% 		 --(\alphacrit:\pv{R}) 
% 		 ($(-upper base)+(360-\alphacrit:\pv{r})$)
% 		 --(360-\alphacrit:\pv{R});
% 		 \path (-upper base) circle[radius=\pv{r}];
		 \begin{scope}
 		  \clip[even odd clip] circle[radius=\pv{R}] [generous outside path];
		  \path[/tikz/3d/hidden]  ($(-upper base)+(\alphacrit:\pv{r})$)
		  arc[start angle=\alphacrit,end  angle=-\alphacrit,radius=\pv{r}];
		  \path[/tikz/3d/visible] (\alphacrit:\pv{R}) --
		  ($(-upper base)+(\alphacrit:\pv{r})$)
		  arc[start angle=\alphacrit,end  angle=360-\alphacrit,radius=\pv{r}]
		  --(360-\alphacrit:\pv{R});
		 \end{scope}
		 \path[/tikz/3d/visible] circle[radius=\pv{R}];
		\fi
	   \else
		\path[/tikz/3d/visible] circle[radius=\pv{R}];	 
		\path[/tikz/3d/hidden] (-upper base) circle[radius=\pv{r}];
	   \fi
   \end{scope}
  \fi
}}}
%		
\pgfmathdeclarefunction{tddistance}{2}{%
\begingroup%
\pgfmathparse@td@FPU{sqrt(TD("#1-#2o#1-#2"))}%
\pgfmathsmuggle\pgfmathresult\endgroup%
}%
% intersection of three spheres (see https://stackoverflow.com/a/18654302)
% 
%     temp1 = P2-P1                                        
%     e_x = temp1/norm(temp1)                              
%     temp2 = P3-P1                                        
%     i = dot(e_x,temp2)                                   
%     temp3 = temp2 - i*e_x                                
%     e_y = temp3/norm(temp3)                              
%     e_z = cross(e_x,e_y)                                 
%     d = norm(P2-P1)                                      
%     j = dot(e_y,temp2)                                   
%     x = (r1*r1 - r2*r2 + d*d) / (2*d)                    
%     y = (r1*r1 - r3*r3 -2*i*x + i*i + j*j) / (2*j)       
%     temp4 = r1*r1 - x*x - y*y                            
%     if temp4<0:                                          
%         raise Exception("The three spheres do not intersect!");
%     z = sqrt(temp4)                                      
%     p_12_a = P1 + x*e_x + y*e_y + z*e_z                  
%     p_12_b = P1 + x*e_x + y*e_y - z*e_z                  
%     return p_12_a,p_12_b                       
\tikzset{3d/.cd,intersection of three spheres/.code={%
\tikzset{3d/aux keys/.cd,#1,/tikz/3d/define orthonormal dreibein}%
\pgfmathsetmacro\pgfutil@tmpA{TD("\pgfkeysvalueof{/tikz/3d/aux keys/A}")}%
\pgfmathsetmacro\pgfutil@tmpB{TD("\pgfkeysvalueof{/tikz/3d/aux keys/B}")}%
\pgfmathsetmacro\pgfutil@tmpC{TD("\pgfkeysvalueof{/tikz/3d/aux keys/C}")}%
\pgfmathsetmacro\pgfutil@tmpex{TD("\pgfkeysvalueof{/tikz/3d/aux keys/ex}")}%
\pgfmathsetmacro\pgfutil@tmpey{TD("\pgfkeysvalueof{/tikz/3d/aux keys/ey}")}%
\pgfmathsetmacro\pgfutil@tmpez{TD("\pgfkeysvalueof{/tikz/3d/aux keys/ez}")}%
\pgfmathsetmacro\pgfutil@tmpAB{TD("(\pgfutil@tmpB)-(\pgfutil@tmpA)")}%
\pgfmathsetmacro\pgfutil@tmpd{sqrt(TD("(\pgfutil@tmpAB)o(\pgfutil@tmpAB)"))}%
\ifdim\pgfutil@tmpd pt<0.1pt\relax
\PackageWarning{3dtools}{Points \pgfkeysvalueof{/tikz/3d/aux keys/A} and \pgfkeysvalueof{/tikz/3d/aux keys/B} are too close to each other.}%
\else
\pgfmathsetmacro\pgfutil@tmpAC{TD("(\pgfutil@tmpC)-(\pgfutil@tmpA)")}% temp2
%     i = dot(e_x,temp2)                                   
\pgfmathsetmacro\pgfutil@tmpi{TD("(\pgfutil@tmpex)o(\pgfutil@tmpAC)")}%
%     j = dot(e_y,temp2)                                   
\pgfmathsetmacro\pgfutil@tmpj{TD("(\pgfutil@tmpey)o(\pgfutil@tmpAC)")}%
\edef\pgfutil@tmprA{\pgfkeysvalueof{/tikz/3d/aux keys/rA}}%
\edef\pgfutil@tmprB{\pgfkeysvalueof{/tikz/3d/aux keys/rB}}%
\edef\pgfutil@tmprC{\pgfkeysvalueof{/tikz/3d/aux keys/rC}}%
%     x = (r1*r1 - r2*r2 + d*d) / (2*d)                    
\pgfmathsetmacro\pgfutil@tmpx{(pow(\pgfutil@tmprA,2)-pow(\pgfutil@tmprB,2)%
+pow(\pgfutil@tmpd,2))/(2*(\pgfutil@tmpd))}%
%     y = (r1*r1 - r3*r3 -2*i*x + i*i + j*j) / (2*j)       
\pgfmathsetmacro\pgfutil@tmpy{(\pgfutil@tmprA*\pgfutil@tmprA-\pgfutil@tmprC*\pgfutil@tmprC%
-2*(\pgfutil@tmpi)*(\pgfutil@tmpx)+(\pgfutil@tmpi)*(\pgfutil@tmpi)%
+(\pgfutil@tmpj)*(\pgfutil@tmpj))/(2*(\pgfutil@tmpj))}%
%     temp4 = r1*r1 - x*x - y*y                            
\pgfmathsetmacro\pgfutil@tmpc{\pgfutil@tmprA*\pgfutil@tmprA-pow(\pgfutil@tmpx,2)-pow(\pgfutil@tmpy,2)}%
\ifdim\pgfutil@tmpc pt<0pt\relax
\PackageWarning{3dtools}{Not all spheres intersect.}%
\else
\pgfmathsetmacro\pgfutil@tmpz{sqrt(\pgfutil@tmpc)}%
\pgfmathsetmacro\pgfutil@tmpd{TD("(\pgfutil@tmpA)+\pgfutil@tmpx*(\pgfutil@tmpex)%
+\pgfutil@tmpy*(\pgfutil@tmpey)+\pgfutil@tmpz*(\pgfutil@tmpez)")}%
\pgfmathsetmacro\pgfutil@tmpe{TD("(\pgfutil@tmpA)+\pgfutil@tmpx*(\pgfutil@tmpex)%
+\pgfutil@tmpy*(\pgfutil@tmpey)-\pgfutil@tmpz*(\pgfutil@tmpez)")}%
\tikzset{insert path={%
(\pgfutil@tmpd) coordinate(\pgfkeysvalueof{/tikz/3d/aux keys/i1}) 
(\pgfutil@tmpe) coordinate(\pgfkeysvalueof{/tikz/3d/aux keys/i2}) 
}}
\fi
\fi
}}%
%
\newif\if@tikz@td@complete@dashes
\@tikz@td@complete@dashesfalse
\tikzset{cheating dash/.code args={on #1 off #2}{% from https://tex.stackexchange.com/a/133357
        % Use csname so catcode of @ doesn't have do be changed.		
        \csname tikz@addoption\endcsname{%
            \pgfgetpath\currentpath%
            \pgfprocessround{\currentpath}{\currentpath}%
            \csname pgf@decorate@parsesoftpath\endcsname{\currentpath}{\currentpath}%
            \pgfmathparse{\csname pgf@decorate@totalpathlength\endcsname-#1}\let\rest=\pgfmathresult%
            \pgfmathparse{#1+#2}\let\onoff=\pgfmathresult%
            \pgfmathparse{max(floor(\rest/\onoff), 1)}\let\nfullonoff=\pgfmathresult%
            \pgfmathparse{max((\rest-\onoff*\nfullonoff)/\nfullonoff+#2, #2)}\let\offexpand=\pgfmathresult%
            \pgfsetdash{{#1}{\offexpand}}{0pt}}%
    },cheating dash/.default={on 2pt off 2pt}}%
\tikzset{3d/.cd,polyhedron/.cd,
face edges/.style={},
edges have complete dashes/.is if=@tikz@td@complete@dashes,
edges have complete dashes/.default=true,
complete dashes/.code args={on #1 off #2}{\tikzset{3d/.cd,polyhedron/.cd,
edges have complete dashes=true,
face edges/.append style={cheating dash=on #1 off #2}}},
complete dashes/.default={on 2pt off 2pt},
draw face with corners/.code={\begingroup\c@pgf@counta0\relax
\pgfutil@for\pgfutil@tmpa:={#1}\do{%
\pgfmathsetmacro\pgfutil@tmpT{TD("\pgfutil@tmpa")}%
\ifcase\c@pgf@counta 
\pgfmathsetmacro\pgfutil@tmpA{TD("\pgfutil@tmpa")}%
\or
\pgfmathsetmacro\pgfutil@tmpB{TD("\pgfutil@tmpa")}%
\or
\pgfmathsetmacro\pgfutil@tmpC{TD("\pgfutil@tmpa")}%
\fi
\advance\c@pgf@counta by1\relax
\ifnum\c@pgf@counta=1\relax
\edef\pgfutil@tmpl{(\pgfutil@tmpT)}% regular list
\edef\pgfutil@tmplf{(\pgfutil@tmpT)}% first
\edef\pgfutil@tmple{(\pgfutil@tmpT)}% edge list
\else
\edef\pgfutil@tmpl{\pgfutil@tmpl -- (\pgfutil@tmpT)}%
\edef\pgfutil@tmple{\pgfutil@tmple\space edge (\pgfutil@tmpT)
(\pgfutil@tmpT)}%
\fi
}%
\ifnum\c@pgf@counta<3\relax
\PackageWarning{3dtools}{A face needs at least three vertices. However, only \the\c@pgf@counta\space vertices were specified.}%
\else
\edef\pgfutil@tmpO{\pgfkeysvalueof{/tikz/3d/polyhedron/O}}%
\pgfmathsetmacro\pgfutil@tmpO{TD("\pgfutil@tmpO")}%
\edef\pgfutil@tmpL{\pgfkeysvalueof{/tikz/3d/polyhedron/L}}%
\pgfmathsetmacro\pgfutil@tmpL{TD("\pgfutil@tmpL")}%
\pgfmathsetmacro{\pgfutil@tmpb}{TD("(\pgfutil@tmpB)-(\pgfutil@tmpA)x(\pgfutil@tmpC)-(\pgfutil@tmpA)")}% 
\pgfmathsetmacro{\pgfutil@tmpc}{TD("(\pgfutil@tmpA)-(\pgfutil@tmpO)")}% 
\pgfmathtruncatemacro{\pgfutil@tmpd}{sign(TD("(\pgfutil@tmpb)o(\pgfutil@tmpc)"))}%
\ifnum\pgfutil@tmpd=-1\relax
\pgfmathsetmacro{\pgfutil@tmpb}{TD("(\pgfutil@tmpB)-(\pgfutil@tmpA)x(\pgfutil@tmpA)-(\pgfutil@tmpC)")}% 
\fi
\pgfmathsetmacro{\pgfutil@tmpe}{screendepth(\pgfutil@tmpb)}%
\pgfmathsetmacro{\pgfutil@tmpf}{sqrt(TD("(\pgfutil@tmpb)o(\pgfutil@tmpb)"))}%
\pgfmathsetmacro{\pgfutil@tmpg}{sqrt(TD("(\pgfutil@tmpL)o(\pgfutil@tmpL)"))}%
\pgfmathsetmacro{\pgfutil@tmph}{TD("(\pgfutil@tmpL)o(\pgfutil@tmpb)")/\pgfutil@tmpf/\pgfutil@tmpg}%
\pgfmathtruncatemacro{\pgfutil@tmpi}{70+30*\pgfutil@tmph}%
\pgfmathtruncatemacro{\pgfutil@tmpi}{\pgfkeysvalueof{/tikz/3d/polyhedron/shading function}(\pgfutil@tmph)}%
\ifdim\pgfutil@tmpe pt<0pt\relax
\begin{pgfonlayer}{\pgfkeysvalueof{/tikz/3d/polyhedron/back layer}}
\tikzset{3d/polyhedron/before hidden,3d/polyhedron/back}%
\if@tikz@td@complete@dashes
\path[fill=tikz@td@face@color!\pgfutil@tmpi!black,3d/polyhedron/back,draw=none] 
	\pgfutil@tmpl -- cycle;
\edef\pgfutil@tmplg{\pgfutil@tmple\space edge \pgfutil@tmplf}%
\draw[3d/polyhedron/back,every edge/.append style={/tikz/3d/polyhedron/face edges}]
	\pgfutil@tmplg;
\else
\draw[fill=tikz@td@face@color!\pgfutil@tmpi!black,3d/polyhedron/back] \pgfutil@tmpl -- cycle;
\fi
\tikzset{3d/polyhedron/after hidden}%
\end{pgfonlayer}
\else
\begin{pgfonlayer}{\pgfkeysvalueof{/tikz/3d/polyhedron/fore layer}}
\tikzset{3d/polyhedron/before visible,3d/polyhedron/fore}%
\if@tikz@td@complete@dashes
\path[fill=tikz@td@face@color!\pgfutil@tmpi!black,3d/polyhedron/fore,draw=none]
 \pgfutil@tmpl -- cycle;
\edef\pgfutil@tmplg{\pgfutil@tmple\space edge \pgfutil@tmplf}%
\draw[3d/polyhedron/fore,every edge/.append style={/tikz/3d/polyhedron/face edges}]
	\pgfutil@tmplg;
\else
\draw[fill=tikz@td@face@color!\pgfutil@tmpi!black,3d/polyhedron/fore] \pgfutil@tmpl -- cycle;
\fi
\tikzset{3d/polyhedron/after visible}%
\end{pgfonlayer}
\fi
\fi
\endgroup
},
color/.code={\colorlet{tikz@td@face@color}{#1}},color=yellow,
O/.initial={(0,0,0)},% point inside the polyhedron
L/.initial={(1,1,1)},% "light source",
fore/.style={draw,solid},fore layer/.initial=main,
back/.style={draw,dashed,fill opacity=0},back layer/.initial=main,
shading function/.initial={tikztdpolyhedronshade},
/tikz/declare function={tikztdpolyhedronshade(\x)=70+30*\x;},
before visible/.code={},after visible/.code={},
before hidden/.code={},after hidden/.code={},
}%
% y cylinder, not yet in the manual
\tikzset{pics/ycylinder/.style={code={
\tikzset{3d/.cd,#1}
\def\pv##1{\pgfkeysvalueof{/tikz/3d/##1}}
\pgfmathsetmacro{\vmin}{atan2(sin(\pgfkeysvalueof{/tikz/3d/theta})*sin(\pgfkeysvalueof{/tikz/3d/phi}),%
min(cos(\pgfkeysvalueof{/tikz/3d/phi}),-1/sqrt(2))*cos(\pgfkeysvalueof{/tikz/3d/theta}))}
\pgfmathsetmacro{\vmax}{\vmin-180}
\path[3d/cylinder/mantle]
  let \p1=($(0,1,0)-(0,0,0)$),\n1={atan2(\y1,\x1)+180} in
  [shading angle=\n1]
  plot[variable=\t,domain=\vmin:\vmax,smooth]
    ({\pv{r}*cos(\t)},0,{\pv{r}*sin(\t)})
    -- 
    plot[variable=\t,domain=\vmax:\vmin,smooth]
    ({\pv{r}*cos(\t)},\pv{h},{\pv{r}*sin(\t)})
    --cycle;
\pgfmathtruncatemacro{\itest}{sign(cos(\pgfkeysvalueof{/tikz/3d/phi}))}
\ifnum\itest=-1
    \path[3d/cylinder/top] plot[variable=\t,domain=0:360,smooth cycle]
    ({\pv{r}*cos(\t)},\pv{h},{\pv{r}*sin(\t)}) ;
\fi
\ifnum\itest=1
    \path[3d/cylinder/top] plot[variable=\t,domain=0:360,smooth cycle]
    ({\pv{r}*cos(\t)},0,{\pv{r}*sin(\t)}) ;
\fi
}},3d/.cd,cylinder/.cd,
mantle/.style={draw},top/.style={draw}}
\tikzset{3d/.cd,visible/.style={draw,solid},
	hidden/.style={draw,very thin,cheating dash},
	r/.initial=1,h/.initial=1,R/.initial=2}
%%
%% decorations
%% 
%% path restoration
\newcounter{tikz@td@recorded@point}
\pgfmathdeclarefunction{tikztdindexoflastcoordinate}{0}{%
\edef\pgfmathresult{\number\value{tikz@td@recorded@point}}%
}%

\tikzset{stored path/.cd,coordinate prefix/.initial={stored-},
	reset coordinate index/.code={\setcounter{tikz@td@recorded@point}{0}},
	restore path/.style={insert path/.expanded={\csname tikz@td@recorded@path@f@#1\endcsname}},%	
	restore reversed path/.style={insert path/.expanded={\csname tikz@td@recorded@path@r@#1\endcsname}},%	
	first coordinate of/.style={insert path/.expanded={(\pgfkeysvalueof{/tikz/stored path/coordinate prefix}#1-0)}},%	
	last coordinate of/.style={insert path/.expanded={(\pgfkeysvalueof{/tikz/stored path/coordinate prefix}#1-\the\numexpr\value{tikz@td@recorded@point})}},%	
	coordinate with index/.style args={#1 of #2}{insert path/.expanded={(\pgfkeysvalueof{/tikz/stored path/coordinate prefix}#2-#1)}},%	
/tikz/store path/.style={/tikz/stored path/reset coordinate index,%
    decorate,decoration={show path construction,
    moveto code={\def\tikz@td@prefix{\pgfkeysvalueof{/tikz/stored path/coordinate prefix}#1-}%
	\path (\tikzinputsegmentfirst) 
		coordinate (\tikz@td@prefix\the\numexpr\value{tikz@td@recorded@point});
    \ifnum\value{tikz@td@recorded@point}=0
	 \expandafter\xdef\csname tikz@td@recorded@path@f@#1\endcsname{% forward
	 	(\tikz@td@prefix\the\numexpr\value{tikz@td@recorded@point})}%
	 \expandafter\xdef\csname tikz@td@recorded@path@r@#1\endcsname{% reverse
	 	(\tikz@td@prefix\the\numexpr\value{tikz@td@recorded@point})}%
	\else
	 \expandafter\xdef\csname tikz@td@recorded@path@f@#1\endcsname{% forward
	 	\csname tikz@td@recorded@path@f@#1\endcsname
		(\tikz@td@prefix\the\numexpr\value{tikz@td@recorded@point})}%
	 \expandafter\xdef\csname tikz@td@recorded@path@r@#1\endcsname{% reverse
	 	(\tikz@td@prefix\the\numexpr\value{tikz@td@recorded@point}) 
		\csname tikz@td@recorded@path@r@#1\endcsname}%
	\fi
	\stepcounter{tikz@td@recorded@point}%	
    },
    lineto code={\def\tikz@td@prefix{\pgfkeysvalueof{/tikz/stored path/coordinate prefix}#1-}%
	\path (\tikzinputsegmentfirst) 
		coordinate (\tikz@td@prefix\the\numexpr\value{tikz@td@recorded@point})
		 (\tikzinputsegmentlast) 
		coordinate (\tikz@td@prefix\the\numexpr\value{tikz@td@recorded@point}+1);
    \ifnum\value{tikz@td@recorded@point}=0
	 \expandafter\xdef\csname tikz@td@recorded@path@f@#1\endcsname{% forward
	 	(\tikz@td@prefix\the\numexpr\value{tikz@td@recorded@point})
		--(\tikz@td@prefix\the\numexpr\value{tikz@td@recorded@point}+1)}%
	 \expandafter\xdef\csname tikz@td@recorded@path@r@#1\endcsname{% reverse
	 	(\tikz@td@prefix\the\numexpr\value{tikz@td@recorded@point}+1)
		--(\tikz@td@prefix\the\numexpr\value{tikz@td@recorded@point})}%
	\else
	 \expandafter\xdef\csname tikz@td@recorded@path@f@#1\endcsname{% forward
	 	\csname tikz@td@recorded@path@f@#1\endcsname
		--(\tikz@td@prefix\the\numexpr\value{tikz@td@recorded@point}+1)}%
	 \expandafter\xdef\csname tikz@td@recorded@path@r@#1\endcsname{% reverse
	 	(\tikz@td@prefix\the\numexpr\value{tikz@td@recorded@point}+1) --
		\csname tikz@td@recorded@path@r@#1\endcsname}%
	\fi
	\stepcounter{tikz@td@recorded@point}%	
    },
    curveto code={\def\tikz@td@prefix{\pgfkeysvalueof{/tikz/stored path/coordinate prefix}#1-}%
      \path (\tikzinputsegmentfirst)
	    coordinate (\tikz@td@prefix\the\numexpr\value{tikz@td@recorded@point})
        (\tikzinputsegmentsupporta) 
		coordinate (\tikz@td@prefix\the\numexpr\value{tikz@td@recorded@point}+1)
		(\tikzinputsegmentsupportb)
		coordinate (\tikz@td@prefix\the\numexpr\value{tikz@td@recorded@point}+2)
        (\tikzinputsegmentlast)
		coordinate (\tikz@td@prefix\the\numexpr\value{tikz@td@recorded@point}+3);
    \ifnum\value{tikz@td@recorded@point}=0
	 \expandafter\xdef\csname tikz@td@recorded@path@f@#1\endcsname{% forward
	 	(\tikz@td@prefix\the\numexpr\value{tikz@td@recorded@point})
		.. controls (\tikz@td@prefix\the\numexpr\value{tikz@td@recorded@point}+1)
		and (\tikz@td@prefix\the\numexpr\value{tikz@td@recorded@point}+2)
		.. (\tikz@td@prefix\the\numexpr\value{tikz@td@recorded@point}+3)}%
	 \expandafter\xdef\csname tikz@td@recorded@path@r@#1\endcsname{% reverse
	 	(\tikz@td@prefix\the\numexpr\value{tikz@td@recorded@point}+3)
		.. controls (\tikz@td@prefix\the\numexpr\value{tikz@td@recorded@point}+2)
		and (\tikz@td@prefix\the\numexpr\value{tikz@td@recorded@point}+1)
		.. (\tikz@td@prefix\the\numexpr\value{tikz@td@recorded@point})}%
	\else
	 \expandafter\xdef\csname tikz@td@recorded@path@f@#1\endcsname{% forward
	 	\csname tikz@td@recorded@path@f@#1\endcsname
		.. controls (\tikz@td@prefix\the\numexpr\value{tikz@td@recorded@point}+1)
		and (\tikz@td@prefix\the\numexpr\value{tikz@td@recorded@point}+2)
		.. (\tikz@td@prefix\the\numexpr\value{tikz@td@recorded@point}+3)}%
	 \expandafter\xdef\csname tikz@td@recorded@path@r@#1\endcsname{% reverse
	 	(\tikz@td@prefix\the\numexpr\value{tikz@td@recorded@point}+3)
		.. controls (\tikz@td@prefix\the\numexpr\value{tikz@td@recorded@point}+2)
		and (\tikz@td@prefix\the\numexpr\value{tikz@td@recorded@point}+1) ..
		\csname tikz@td@recorded@path@r@#1\endcsname}%
	\fi
	\stepcounter{tikz@td@recorded@point}%	
	\stepcounter{tikz@td@recorded@point}%	
	\stepcounter{tikz@td@recorded@point}%	
    },
    closepath code={\def\tikz@td@prefix{\pgfkeysvalueof{/tikz/stored path/coordinate prefix}#1-}%
	\path (\tikzinputsegmentfirst) 
		coordinate (\tikz@td@prefix\the\numexpr\value{tikz@td@recorded@point})
		 (\tikzinputsegmentlast) 
		coordinate (\tikz@td@prefix\the\numexpr\value{tikz@td@recorded@point}+1);
    \ifnum\value{tikz@td@recorded@point}=0
	 \expandafter\xdef\csname tikz@td@recorded@path@f@#1\endcsname{% forward
	 	(\tikz@td@prefix\the\numexpr\value{tikz@td@recorded@point})
		--(\tikz@td@prefix\the\numexpr\value{tikz@td@recorded@point}+1)}%
	 \expandafter\xdef\csname tikz@td@recorded@path@r@#1\endcsname{% reverse
	 	(\tikz@td@prefix\the\numexpr\value{tikz@td@recorded@point}+1)
		--(\tikz@td@prefix\the\numexpr\value{tikz@td@recorded@point})}%
	\else
	 \expandafter\xdef\csname tikz@td@recorded@path@f@#1\endcsname{% forward
	 	\csname tikz@td@recorded@path@f@#1\endcsname
		--(\tikz@td@prefix\the\numexpr\value{tikz@td@recorded@point}+1)}%
	 \expandafter\xdef\csname tikz@td@recorded@path@r@#1\endcsname{% reverse
	 	(\tikz@td@prefix\the\numexpr\value{tikz@td@recorded@point}+1) --
		\csname tikz@td@recorded@path@r@#1\endcsname}%
	\fi
	  } }}}
%% coil
\newif\ifcoil@closed%
\pgfkeys{%%
/pgf/decoration/.cd,%
3d coil color/.initial=black,%
3d coil width/.initial=0.4pt,%
3d coil dist/.initial=0.6pt,%
3d coil opacity/.initial=1,%
3d coil closed/.code=\coil@closedtrue%
}%
% https://tex.stackexchange.com/a/219088/121799%
\tikzset{get stroke color/.code={%%
    \expandafter\global% Jump over, now we have \global%
    \expandafter\let% Jump over now we have \global\let%
    \expandafter\pgfsavedstrokecolor% Jump we have \global\let\pgf...%
    \csname\string\color@pgfstrokecolor\endcsname% Finally expand this and put it at the end %
    },                                           % \global\let\pgf...{} in expanded form %
    restore stroke color/.code={\pgf@setstrokecolor#1},%
}%
\def\pgfpoint@onthreedcoil#1#2#3{%%
  \pgf@x=#1\pgfdecorationsegmentamplitude%%
  \pgf@x=\pgfdecorationsegmentaspect\pgf@x%%
  \pgf@y=#2\pgfdecorationsegmentamplitude%%
  \pgf@xa=0.083333333333\pgfdecorationsegmentlength%%
  \advance\pgf@x by#3\pgf@xa%%
  \advance\pgf@x by-\generaloffset pt%%
}%
% coil decoration%
%%
% Parameters: \pgfdecorationsegmentamplitude, \pgfdecorationsegmentlength,%
\pgfdeclaredecoration{3d complete coil}{initial}%
{ %
    \state{initial}[width=0.5*\pgfdecorationsegmentlength,%
    next state=coil, persistent precomputation={% from https://tex.stackexchange.com/a/25689/121799%
    \pgfmathsetmacro\matchinglength{\pgfdecoratedinputsegmentlength / int(\pgfdecoratedinputsegmentlength/\pgfdecorationsegmentlength)}%
    \setlength{\pgfdecorationsegmentlength}{\matchinglength pt}%
    \tikzset{get stroke color}%
    \pgfmathsetmacro{\generaloffset}{\pgfdecorationsegmentlength}%
    \pgfmathsetmacro{\initialoffset}{1.5*\pgfdecorationsegmentlength}%
    \pgfmathsetmacro{\auxoffset}{2.5*\pgfdecorationsegmentlength}%
  }]    { %
    % line in the back%
    %%
	\edef\TDCoilColor{\pgfkeysvalueof{/pgf/decoration/3d coil color}}%
	\edef\TDCoilOpacity{\pgfkeysvalueof{/pgf/decoration/3d coil opacity}}%
	\edef\TDCoilWidth{\pgfkeysvalueof{/pgf/decoration/3d coil width}}%
	\edef\TDCoilDist{\pgfkeysvalueof{/pgf/decoration/3d coil dist}}%
    \pgfsetstrokecolor{\pgfkeysvalueof{/pgf/decoration/3d coil color}}%
    \pgfsetfillcolor{\pgfkeysvalueof{/pgf/decoration/3d coil color}}%
    \pgfsetstrokeopacity{\pgfkeysvalueof{/pgf/decoration/3d coil opacity}}%
    \pgfsetlinewidth{\pgfkeysvalueof{/pgf/decoration/3d coil width}}     %
    \ifcoil@closed%
     \begingroup%
      \def\generaloffset{\auxoffset}%
      \pgfpathmoveto{\pgfpoint@onthreedcoil{1    }{ 1    }{15}}%
      \pgfpathcurveto%
      {\pgfpoint@onthreedcoil{1.555}{ 1    }{16}}%
      {\pgfpoint@onthreedcoil{2    }{ 0.555}{17}}%
      {\pgfpoint@onthreedcoil{2    }{ 0    }{18}}%
      \pgfcoordinate{TD@coilast}{\pgfpoint@onthreedcoil{2    }{ 0    }{18}}%
      \pgfcoordinate{TD@coilfirst}{\pgfpoint@onthreedcoil{1    }{ 1    }{15}}%
      \pgfusepath{stroke} %
      \pgfsetstrokecolor{\pgfkeysvalueof{/pgf/decoration/3d coil color}}%
     \endgroup%
    \fi%
    \begingroup %%
    \def\generaloffset{\initialoffset}%
    \ifcoil@closed%
     \pgfpathmoveto{\pgfpointanchor{TD@coilast}{center}}%
    \else%
     \pgfpathmoveto{\pgfpointorigin}%
    \fi%
    \pgfpathcurveto%
    {\pgfpoint@onthreedcoil{2    }{-0.555}{7}}%
    {\pgfpoint@onthreedcoil{1.555}{-1    }{8}}%
    {\pgfpoint@onthreedcoil{1    }{-1    }{9}}%
    \pgfusepath{stroke} %
    %%
    % white background for front thick part%
    %%
    \pgfsetstrokeopacity{1}%
    \pgfsetstrokecolor{white}%
    \pgfsetfillcolor{white}%
    \pgfsetlinewidth{1.5*\pgfkeysvalueof{/pgf/decoration/3d coil width}+1.5*\pgfkeysvalueof{/pgf/decoration/3d coil dist}}%
    \pgfpathmoveto{\pgfpoint@onthreedcoil{1    }{-1    }{9}}%
    % draw forward%
    \pgfpathcurveto%
    {\pgfpoint@onthreedcoil{0.445}{-1    }{10}}%
    {\pgfpoint@onthreedcoil{0    }{-0.555}{11.25}}%
    {\pgfpoint@onthreedcoil{0    }{ 0    }{12.5}}%
    \pgfpathcurveto%
    {\pgfpoint@onthreedcoil{0    }{ 0.555}{13.25}}%
    {\pgfpoint@onthreedcoil{0.445}{ 1    }{14.25}}%
    {\pgfpoint@onthreedcoil{1    }{ 1    }{15}}%
    % draw the curve back%
    \pgfpathcurveto%
    {\pgfpoint@onthreedcoil{0.445}{ 1    }{14}}%
    {\pgfpoint@onthreedcoil{0    }{ 0.555}{12.75}}%
    {\pgfpoint@onthreedcoil{0    }{ 0    }{11.5}}%
    \pgfpathcurveto%
    {\pgfpoint@onthreedcoil{0    }{-0.555}{10.75}}%
    {\pgfpoint@onthreedcoil{0.445}{-1    }{10}}%
    {\pgfpoint@onthreedcoil{1    }{-1    }{9}}%
    \pgfusepath{stroke,fill} %
    % %
    % draw the thick foreground path%
    %%
    \pgfsetstrokecolor{\pgfkeysvalueof{/pgf/decoration/3d coil color}}%
    \pgfsetfillcolor{\pgfkeysvalueof{/pgf/decoration/3d coil color}}%
    \pgfsetstrokeopacity{\pgfkeysvalueof{/pgf/decoration/3d coil opacity}}%
    \pgfpathmoveto{\pgfpoint@onthreedcoil{1    }{ 1    }{3}}%
    \pgfsetlinewidth{\pgfkeysvalueof{/pgf/decoration/3d coil width}} %
    % forward shifted +%
    \pgfpathmoveto{\pgfpoint@onthreedcoil{1    }{-1    }{9}}%
    \pgfpathcurveto%
    {\pgfpoint@onthreedcoil{0.445}{-1    }{10}}%
    {\pgfpoint@onthreedcoil{0    }{-0.555}{11.25}}%
    {\pgfpoint@onthreedcoil{0    }{ 0    }{12.5}}%
    \pgfpathcurveto%
    {\pgfpoint@onthreedcoil{0    }{ 0.555}{13.25}}%
    {\pgfpoint@onthreedcoil{0.445}{ 1    }{14.25}}%
    {\pgfpoint@onthreedcoil{1    }{ 1    }{15}}%
    % draw the curve back shfted -%
    \pgfpathcurveto%
    {\pgfpoint@onthreedcoil{0.445}{ 1    }{14}}%
    {\pgfpoint@onthreedcoil{0    }{ 0.555}{12.75}}%
    {\pgfpoint@onthreedcoil{0    }{ 0    }{11.5}}%
    \pgfpathcurveto%
    {\pgfpoint@onthreedcoil{0    }{-0.555}{10.75}}%
    {\pgfpoint@onthreedcoil{0.445}{-1    }{10}}%
    {\pgfpoint@onthreedcoil{1    }{-1    }{9}}%
    \pgfusepath{stroke,fill} %
    \pgfpathmoveto{\pgfpoint@onthreedcoil{1    }{ 1    }{15}}%
    \pgfpathcurveto%
    {\pgfpoint@onthreedcoil{1.555}{ 1    }{16}}%
    {\pgfpoint@onthreedcoil{2    }{ 0.555}{17}}%
    {\pgfpoint@onthreedcoil{2    }{ 0    }{18}}%
    \pgfcoordinate{TD@coilast}{\pgfpoint@onthreedcoil{2    }{ 0    }{18}} %
    \pgfusepath{stroke}     %
    \endgroup%
  }%
  \state{coil}[switch if less than=%%
    1.9*\pgfdecorationsegmentlength  to last,%
               width=+\pgfdecorationsegmentlength]%
    { % line in the back%
    %%
    \pgfsetstrokecolor{\pgfkeysvalueof{/pgf/decoration/3d coil color}}%
    \pgfsetfillcolor{\pgfkeysvalueof{/pgf/decoration/3d coil color}}%
    \pgfsetstrokeopacity{\pgfkeysvalueof{/pgf/decoration/3d coil opacity}}%
    \pgfpathmoveto{\pgfpointanchor{TD@coilast}{center}}%
    \pgfsetlinewidth{\pgfkeysvalueof{/pgf/decoration/3d coil width}} %
    \pgfpathcurveto%
    {\pgfpoint@onthreedcoil{2    }{-0.555}{7}}%
    {\pgfpoint@onthreedcoil{1.555}{-1    }{8}}%
    {\pgfpoint@onthreedcoil{1    }{-1    }{9}}%
    \pgfusepath{stroke} %
    %%
    % white background for front thick part%
    %%
    \pgfsetstrokeopacity{1}%
    \pgfsetstrokecolor{white}%
    \pgfsetfillcolor{white}%
    \pgfsetlinewidth{1.5*\pgfkeysvalueof{/pgf/decoration/3d coil width}+1.5*\pgfkeysvalueof{/pgf/decoration/3d coil dist}}%
    \pgfpathmoveto{\pgfpoint@onthreedcoil{1    }{ 1    }{3}}%
    \pgfpathmoveto{\pgfpoint@onthreedcoil{1    }{-1    }{9}}%
    % draw forward%
    \pgfpathcurveto%
    {\pgfpoint@onthreedcoil{0.445}{-1    }{10}}%
    {\pgfpoint@onthreedcoil{0    }{-0.555}{11.25}}%
    {\pgfpoint@onthreedcoil{0    }{ 0    }{12.5}}%
    \pgfpathcurveto%
    {\pgfpoint@onthreedcoil{0    }{ 0.555}{13.25}}%
    {\pgfpoint@onthreedcoil{0.445}{ 1    }{14.25}}%
    {\pgfpoint@onthreedcoil{1    }{ 1    }{15}}%
    % draw the curve back%
    \pgfpathcurveto%
    {\pgfpoint@onthreedcoil{0.445}{ 1    }{14}}%
    {\pgfpoint@onthreedcoil{0    }{ 0.555}{12.75}}%
    {\pgfpoint@onthreedcoil{0    }{ 0    }{11.5}}%
    \pgfpathcurveto%
    {\pgfpoint@onthreedcoil{0    }{-0.555}{10.75}}%
    {\pgfpoint@onthreedcoil{0.445}{-1    }{10}}%
    {\pgfpoint@onthreedcoil{1    }{-1    }{9}}%
    \pgfusepath{stroke,fill} %
    % %
    % draw the thick foreground path%
    %%
    \pgfsetstrokecolor{\pgfkeysvalueof{/pgf/decoration/3d coil color}}%
    \pgfsetfillcolor{\pgfkeysvalueof{/pgf/decoration/3d coil color}}%
    \pgfsetstrokeopacity{\pgfkeysvalueof{/pgf/decoration/3d coil opacity}}%
    \pgfpathmoveto{\pgfpoint@onthreedcoil{1    }{ 1    }{3}}%
    \pgfsetlinewidth{\pgfkeysvalueof{/pgf/decoration/3d coil width}} %
    % forward shifted +%
    \pgfpathmoveto{\pgfpoint@onthreedcoil{1    }{-1    }{9}}%
    \pgfpathcurveto%
    {\pgfpoint@onthreedcoil{0.445}{-1    }{10}}%
    {\pgfpoint@onthreedcoil{0    }{-0.555}{11.25}}%
    {\pgfpoint@onthreedcoil{0    }{ 0    }{12.5}}%
    \pgfpathcurveto%
    {\pgfpoint@onthreedcoil{0    }{ 0.555}{13.25}}%
    {\pgfpoint@onthreedcoil{0.445}{ 1    }{14.25}}%
    {\pgfpoint@onthreedcoil{1    }{ 1    }{15}}%
    % draw the curve back shfted -%
    \pgfpathcurveto%
    {\pgfpoint@onthreedcoil{0.445}{ 1    }{14}}%
    {\pgfpoint@onthreedcoil{0    }{ 0.555}{12.75}}%
    {\pgfpoint@onthreedcoil{0    }{ 0    }{11.5}}%
    \pgfpathcurveto%
    {\pgfpoint@onthreedcoil{0    }{-0.555}{10.75}}%
    {\pgfpoint@onthreedcoil{0.445}{-1    }{10}}%
    {\pgfpoint@onthreedcoil{1    }{-1    }{9}}%
    \pgfusepath{stroke,fill} %
    \pgfpathmoveto{\pgfpoint@onthreedcoil{1    }{ 1    }{15}}%
    \pgfpathcurveto%
    {\pgfpoint@onthreedcoil{1.555}{ 1    }{16}}%
    {\pgfpoint@onthreedcoil{2    }{ 0.555}{17}}%
    {\pgfpoint@onthreedcoil{2    }{ 0    }{18}}%
    \pgfusepath{stroke} %
    \pgfcoordinate{TD@coilast}{\pgfpoint@onthreedcoil{2    }{ 0    }{18}} %
  }%
  \state{last}[next state=final]%
    { % line in the back%
    %%
    \pgfsetstrokecolor{\pgfkeysvalueof{/pgf/decoration/3d coil color}}%
    \pgfsetfillcolor{\pgfkeysvalueof{/pgf/decoration/3d coil color}}%
    \pgfsetstrokeopacity{\pgfkeysvalueof{/pgf/decoration/3d coil opacity}}%
    \pgfpathmoveto{\pgfpointanchor{TD@coilast}{center}}%
    \pgfsetlinewidth{\pgfkeysvalueof{/pgf/decoration/3d coil width}} %
    \pgfpathcurveto%
    {\pgfpoint@onthreedcoil{2    }{-0.555}{7}}%
    {\pgfpoint@onthreedcoil{1.555}{-1    }{8}}%
    {\pgfpoint@onthreedcoil{1    }{-1    }{9}}%
    \pgfusepath{stroke} %
    % %
    % draw the thick foreground path%
    %%
    \ifcoil@closed %\pgfpointanchor{TD@coilfirst}{center}%
     %%
     % white background for front thick part%
     %%
     \pgfsetstrokeopacity{1}%
     \pgfsetstrokecolor{white}%
     \pgfsetfillcolor{white}%
     \pgfsetlinewidth{1.5*\pgfkeysvalueof{/pgf/decoration/3d coil width}+1.5*\pgfkeysvalueof{/pgf/decoration/3d coil dist}}%
     \pgfpathmoveto{\pgfpoint@onthreedcoil{1    }{ 1    }{3}}%
     \pgfpathmoveto{\pgfpoint@onthreedcoil{1    }{-1    }{9}}%
     % draw forward%
     \pgfpathcurveto%
     {\pgfpoint@onthreedcoil{0.445}{-1    }{10}}%
     {\pgfpoint@onthreedcoil{0    }{-0.555}{11.25}}%
     {\pgfpoint@onthreedcoil{0    }{ 0    }{12.5}}%
     \pgfpathcurveto%
     {\pgfpoint@onthreedcoil{0    }{ 0.555}{13.25}}%
     {\pgfpoint@onthreedcoil{0.445}{ 1    }{14.25}}%
     {\pgfpointanchor{TD@coilfirst}{center}}%
     % draw the curve back%
     \pgfpathcurveto%
     {\pgfpoint@onthreedcoil{0.445}{ 1    }{14}}%
     {\pgfpoint@onthreedcoil{0    }{ 0.555}{12.75}}%
     {\pgfpoint@onthreedcoil{0    }{ 0    }{11.5}}%
     \pgfpathcurveto%
     {\pgfpoint@onthreedcoil{0    }{-0.555}{10.75}}%
     {\pgfpoint@onthreedcoil{0.445}{-1    }{10}}%
     {\pgfpoint@onthreedcoil{1    }{-1    }{9}}%
     \pgfusepath{stroke,fill} %
     \pgfsetstrokecolor{\pgfkeysvalueof{/pgf/decoration/3d coil color}}%
     \pgfsetfillcolor{\pgfkeysvalueof{/pgf/decoration/3d coil color}}%
     \pgfsetstrokeopacity{\pgfkeysvalueof{/pgf/decoration/3d coil opacity}}%
     \pgfpathmoveto{\pgfpoint@onthreedcoil{1    }{ 1    }{3}}%
     \pgfsetlinewidth{\pgfkeysvalueof{/pgf/decoration/3d coil width}} %
     % forward shifted +%
     \pgfpathmoveto{\pgfpoint@onthreedcoil{1    }{-1    }{9}}%
     \pgfpathcurveto%
     {\pgfpoint@onthreedcoil{0.445}{-1    }{10}}%
     {\pgfpoint@onthreedcoil{0    }{-0.555}{11.25}}%
     {\pgfpoint@onthreedcoil{0    }{ 0    }{12.5}}%
     \pgfpathcurveto%
     {\pgfpoint@onthreedcoil{0    }{ 0.555}{13.25}}%
     {\pgfpoint@onthreedcoil{0.445}{ 1    }{14.25}}%
     {\pgfpointanchor{TD@coilfirst}{center}}%
     % draw the curve back shifted %
     \pgfpathcurveto%
     {\pgfpoint@onthreedcoil{0.445}{ 1    }{14}}%
     {\pgfpoint@onthreedcoil{0    }{ 0.555}{12.75}}%
     {\pgfpoint@onthreedcoil{0    }{ 0    }{11.5}}%
     \pgfpathcurveto%
     {\pgfpoint@onthreedcoil{0    }{-0.555}{10.75}}%
     {\pgfpoint@onthreedcoil{0.445}{-1    }{10}}%
     {\pgfpoint@onthreedcoil{1    }{-1    }{9}}%
     \pgfusepath{stroke,fill} %
    \else%
     %%
     % white background for front thick part%
     %%
     \pgfsetstrokeopacity{1}%
     \pgfsetstrokecolor{white}%
     \pgfsetfillcolor{white}%
     \pgfsetlinewidth{1.5*\pgfkeysvalueof{/pgf/decoration/3d coil width}+1.5*\pgfkeysvalueof{/pgf/decoration/3d coil dist}}%
     \pgfpathmoveto{\pgfpoint@onthreedcoil{1}{1}{3}}%
     \pgfpathmoveto{\pgfpoint@onthreedcoil{1}{-1}{9}}%
     % draw forward%
     \pgfpathcurveto%
     {\pgfpoint@onthreedcoil{0.445}{-1    }{10}}%
     {\pgfpoint@onthreedcoil{0    }{-0.555}{11.25}}%
     {\pgfpoint@onthreedcoil{0    }{ 0    }{12.5}}%
     \pgfpathcurveto%
     {\pgfpoint@onthreedcoil{0    }{ 0.555}{13.25}}%
     {\pgfpoint@onthreedcoil{0.445}{ 1    }{14.25}}%
     {\pgfpoint@onthreedcoil{1    }{ 1    }{15}}%
     % draw the curve back%
     \pgfpathcurveto%
     {\pgfpoint@onthreedcoil{0.445}{ 1    }{14}}%
     {\pgfpoint@onthreedcoil{0    }{ 0.555}{12.75}}%
     {\pgfpoint@onthreedcoil{0    }{ 0    }{11.5}}%
     \pgfpathcurveto%
     {\pgfpoint@onthreedcoil{0    }{-0.555}{10.75}}%
     {\pgfpoint@onthreedcoil{0.445}{-1    }{10}}%
     {\pgfpoint@onthreedcoil{1    }{-1    }{9}}%
     \pgfusepath{stroke,fill} %
     \pgfsetstrokecolor{\pgfkeysvalueof{/pgf/decoration/3d coil color}}%
     \pgfsetfillcolor{\pgfkeysvalueof{/pgf/decoration/3d coil color}}%
     \pgfsetstrokeopacity{\pgfkeysvalueof{/pgf/decoration/3d coil opacity}}%
     \pgfpathmoveto{\pgfpoint@onthreedcoil{1}{1}{3}}%
     \pgfsetlinewidth{\pgfkeysvalueof{/pgf/decoration/3d coil width}} %
     % forward shifted +%
     \pgfpathmoveto{\pgfpoint@onthreedcoil{1}{-1}{9}}%
     \pgfpathcurveto%
     {\pgfpoint@onthreedcoil{0.445}{-1    }{10}}%
     {\pgfpoint@onthreedcoil{0    }{-0.555}{11.25}}%
     {\pgfpoint@onthreedcoil{0    }{ 0    }{12.5}}%
     \pgfpathcurveto%
     {\pgfpoint@onthreedcoil{0    }{ 0.555}{13.25}}%
     {\pgfpoint@onthreedcoil{0.445}{ 1    }{14.25}}%
     {\pgfpoint@onthreedcoil{1    }{ 1    }{15}}%
     % draw the curve back shifted %
     \pgfpathcurveto%
     {\pgfpoint@onthreedcoil{0.445}{ 1    }{14}}%
     {\pgfpoint@onthreedcoil{0    }{ 0.555}{12.75}}%
     {\pgfpoint@onthreedcoil{0    }{ 0    }{11.5}}%
     \pgfpathcurveto%
     {\pgfpoint@onthreedcoil{0    }{-0.555}{10.75}}%
     {\pgfpoint@onthreedcoil{0.445}{-1    }{10}}%
     {\pgfpoint@onthreedcoil{1    }{-1    }{9}}%
     \pgfusepath{stroke,fill} %
    \fi%
    \pgfpathmoveto{\pgfpoint@onthreedcoil{1    }{ 1    }{15}}%
    \ifcoil@closed %TD@coilfirst%
    \else%
     \pgfpathcurveto%
     {\pgfpoint@onthreedcoil{1.555}{ 1    }{16}}%
     {\pgfpoint@onthreedcoil{2    }{ 0.555}{17}}%
     {\pgfpoint@onthreedcoil{2    }{ 0    }{18}}%
    \fi%
    \pgfusepath{stroke} %
    %\pgfcoordinate{TD@coilast}{\pgfpoint@onthreedcoil{2    }{ 0    }{18}} %
  }%
  \state{final}%
  {%
    \pgfpathmoveto{\pgfpointdecoratedpathlast}%
    \tikzset{restore stroke color/.expand once=\pgfsavedstrokecolor}%
  }%
}%
\makeatother
\endinput
